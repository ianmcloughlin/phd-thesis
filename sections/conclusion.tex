%!TEX root = ../thesis.tex
\chapter*{Conclusion}
In this thesis we have seen that a number of type II codes can be generated by zero divisors in group rings.
The group rings in question are dihedral groups over the finite field with two elements.
The codes have been constructed in much the same way as cyclic codes in the past have been constructed in residue class rings.
Among these codes are many extremal type II codes.

The zero divisors are of the form $1+a \mathbf{f}$ where $a$ is the generator of order two of the group and $\mathbf{f}$ is a sum of powers of the other generator.
The advantages of working with such zero divisors are many.
In chapter \ref{chap:extgolaycode} the zero divisors offered constructions of the code in which it was immediately obvious that the code was self-dual, doubly even and a principal left ideal of the zero divisor in the group ring.
A reverse circulant generator matrix for the code was then given, which generated the code as quasi cyclic of order two.
Using a little algebra it was readily verifiable that the code was of minimum distance eight.

The zero divisors given in chapter \ref{chap:fortyeight} generated the $(48,24,12)$ extremal type II code.
Again the same properties were readily identifiable in this case.
The code was seen to be self-dual, doubly even, quasi cyclic of index two and generated by a reverse circulant matrix.
We saw that the code was of minimum distance at least eight.
The proof that the code was actually of minimum distance twelve was lengthy, but nonetheless algebraic.

Finally in chapter \ref{chap:longer} we discussed the more general case of constructing a type II code in a dihedral group ring.
Generators of numerous type II codes were given, each submitting to the general proofs that such codes were self-dual, doubly even, quasi cyclic of index two, reverse circulantly generated and of minimum distance at least eight.
Properties of the codes were discussed that allow the rapid calculation of such codes minimum distance by computer.
Some examples of type I codes were also given that submitted readily to many of the proofs.

We hope to have convinced the reader by this stage that the use of the dihedral group rings in the construction of self-dual error correcting codes warrants further research.

\begin{flushleft}
Thank you for reading.
\end{flushleft}

