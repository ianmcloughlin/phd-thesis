%!TEX root = ../thesis.tex
\chapter{The $(24,12,8)$ Extended Binary Golay Code}
\label{chap:extgolaycode}
In this chapter we give a new construction of the $(24,12,8)$ extended binary Golay code.
It is constructed from a zero divisor in a group ring over a dihedral group.
The construction resembles that of a cyclic code using a polynomial.

The $(24,12,8)$ code is the extension of the $(23,12,7)$ binary Golay code by a single even parity bit.
The $(23,12,7)$ code is a well known code appearing extensively in coding theory literature.
It has a number of properties that are of importance in both mathematics and coding theory.
These are demonstrated and proven in most introductory books on coding theory~\cite{huf03}.
We will only examine the extended code here.
While it does not share many of the $(23,12,7)$ code's properties, it does exhibit other significant traits.
These have been previously explored in other efforts.
However, they are readily exposed by the new construction given here.

In the first part of this chapter we show that twenty-four different zero divisors in a given dihedral group ring can be used to construct the code.
These zero divisors were originally found by computer search.
We prove that we have in fact constructed the code, showing that it is self-dual, doubly even, of dimension twelve and of minimum distance eight.
We further show that the code is in a fact an ideal in the group ring and that it is quasi cyclic.
These results have already been published in the Institute of Electrical and Electronic Engineer's journal ``Transactions on Information Theory''~\cite{mcl08}.
In the second part of the chapter we prove that these zero divisors are the only ones of their kind in their group ring capable of generating the code.
We start with a brief history of the binary Golay codes.


\section{History of the Code}
The extended binary Golay code is the unique $(24,12,8)$ linear block code up to equivalence~\cite[p.~401]{huf03}.
The code is interesting from both a coding theory and a mathematical perspective as it is an extremal type II code.
A type II code is a binary linear block code that is both self-dual and doubly even\footnote{A doubly even code is one in which the weight of each codeword is congruent to zero modulo four.}~\cite[p.~339]{huf03}.
An \emph{extremal} type II code is a type II code with minimum distance $(4\lfloor n/24 \rfloor + 4)$ where $n$ is the code's length~\cite[p.~346]{huf03}.
This is the best possible minimum distance for a binary type II code of length $n$~\cite[p.~344]{huf03}.

Thus eight is the best minimum distance a binary type II code of length twenty-four can achieve.
It is in fact the best possible minimum distance a binary code of length twenty-four and dimension twelve can have, as evidenced by the Griesmer bound~\cite[p.~81]{huf03}:
\[ n \ge \sum_{i=0}^{k-1} \bigg{\lceil} \frac{d}{2^i} \bigg{\rceil} \]
where $n$ is the length of the code, $k$ the dimension and $d$ the minimum distance.

The (non-extended) $(23,12,7)$ binary Golay code is interesting for many reasons.
It was originally constructed by Marcel J. E. Golay using Pascal's triangle.
He published this result in 1949~\cite{gol49}.
The code turned out to be cyclic and can be generated by the polynomial $1+x+x^5+x^6+x^7+x^9+x^{11}$~\cite[p.~401]{huf03}.
It is also a perfect binary linear block code, meaning it exactly meets the sphere packing bound:
\[\sum_{i=0}^{ \lfloor d-1/2 \rfloor } {n \choose i} \leq 2^{n-k}\]
where $n$ is the length of the code, $k$ the dimension and $d$ the minimum distance~\cite[p.~48]{huf03}.
Thus:
\[\sum_{i=0}^{3} {23 \choose i} = 2^{11}.\]
In essence the code has the maximum number of codewords that a binary linear code of length twenty-three and distance seven can have~\cite[p.~49]{huf03}.
In fact it is the only non-trivial, multiple-error correcting binary code that is perfect other than the Hamming codes~\cite[p.~49]{huf03}.
Furthermore it is the unique binary code of its length, dimension and minimum distance up to equivalence~\cite[p.~49]{huf03}.

The properties discussed above for each of the two codes are not shared by the other.
The extended $(24,12,8)$ code is not perfect.
Nor is it cyclic since a doubly even self-dual code cannot be cyclic~\cite{slo83}.
On the other hand the $(23,12,7)$ code is not self-dual.
It is of dimension twelve but its null space is of dimension (twenty-three minus twelve equals) eleven.
Moreover the $(23,12,7)$ code is not singly even---let alone doubly even---since its minimum distance is odd.

Thus both codes are of interest in their own right.
Much is known about the $(23,12,7)$ code due to its cyclic nature.
It can be constructed from a single polynomial, it is an ideal in a residue class ring and its minimum distance can be calculated relatively quickly.

In the following chapter we prove that the extended code is also constructible using a single polynomial-like generating element, that it is an ideal in a group ring and that its minimum distance is readily calculated.
We'll start now by discussing some elements of interest in a given dihedral group ring.


\section{A Group Ring Code}
\label{sect:listgolayzds}
The first part of this chapter is dedicated to showing that the $(24,12,8)$ extended binary Golay code can be generated by a zero divisor in the group ring $\mathbb{Z}_2 \mathbf{D}_{24}$.
The group ring $\mathbb{Z}_2 \mathbf{D}_{24}$ is formed by the finite field with two elements $\mathbb{Z}_2$ and the dihedral group of order twenty-four $\mathbf{D}_{24}$.
In accordance with the previous chapter we denote the generators of the group by $a$ and $b$.
The generator $a$ is that of order two and $b$ is the generator of order twelve.

Our motivation for using this group ring may not be obvious.
Our reason for using the finite field with two elements is straightforward.
The components of the codewords in a group ring code come from the underlying ring.
The code is binary, and hence its components are elements of $\mathbb{Z}_2$.

The dihedral group is a less obvious choice.
The length of the $(24,12,8)$ extended binary Golay code is twenty-four.
The length of a group ring code is the order of the underlying group.
Thus for the code to be a group ring code the group involved would be of order twenty-four.
There are many groups of order twenty-four however.

In the previous chapter we saw that the dihedral group of order eight generated the extended Hamming $(8,4,4)$ code.
This code shares many of the extended binary Golay code's properties.
Both are self-dual and doubly even.
They are also extensions of perfect cyclic codes.
This led to our investigation of the dihedral group of order twenty-four.
We were successful in our attempts to find generating elements for the code using it.

In total we found twenty-four zero divisors of the form $1+a\mathbf{f}$ in $\mathbb{Z}_2 \mathbf{D}_{24}$ that generate the code where $\mathbf{f}$ is a sum of powers of the group element $b$.
The GUAVA package in GAP was used\cite{gap06}. 
The occurrence of the powers of $b$ with non-zero coefficients in $\mathbf{f}$ is the only difference between the zero divisors.
We chose elements of this form because their group ring matrices have some useful properties, as we will see later in the chapter.

Throughout this chapter we will use one of the zero divisors, $u = 1+a\mathbf{f} = 1 + a(b + b^2 + b^4 + b^5 + b^6 + b^7 + b^9)$, as an example.
The other twenty-three zero divisors are of the form $1+ab^i\mathbf{f}$ for $i$ from one to eleven and $1+ab^j\mathbf{f}^{\textrm{T}}$ for $j$ from zero to eleven.
Trivial changes to the arguments in this chapter will adapt them to work for any of the other twenty-three zero divisors.
In any case we will see later that the generator matrices derived from any of these other twenty-three elements are permutation equivalent to that derived here for $u$.

The code, as we will see, is the principal left ideal of $u$ in $\mathbb{Z}_2 \mathbf{D}_{24}$.
Thus the code is constructed in group ring form as $\mathbb{Z}_2 \mathbf{D}_{24} u$.
The row space of the group ring matrix of $u$, as described in the previous chapter, is the vector form of the code.
In the following sections we rely on the interplay between the vector and group ring forms of the code to prove the results.
We describe the group ring matrix of $u$ next.

\section{The Group Ring Matrix}
\label{sect:golayrgmatrix}
Let us now construct the group ring matrix of the element $u = 1 + a(b + b^2 + b^4 + b^5 + b^6 + b^7 + b^9)$ in $\mathbb{Z}_2 \mathbf{D}_{24}$.
We use the listing $\{1,b,b^2,\ldots,b^{11},a,ab,ab^2,\ldots,ab^{11}\}$ of  $\mathbf{D}_{24}$.
The reason for choosing this listing is that the group ring matrices of the elements of the form $1+a\mathbf{f}$ have some useful properties under its influence.
According to the listing the group ring matrix $U$ of $u$ is of the form:
\[\left[ \begin{array}{c|c}
I & A \\
\hline
A & I
\end{array} \right] \]
where $I$ is the twelve by twelve identity matrix and $A$ is the twelve by twelve matrix in figure \ref{fig:golayAmat}.
Note that this is the same form of matrix given in section \ref{sect:dihedralrgmatrices}, albeit with larger submatrices.
\begin{figure}
\begin{center}
\[ \left[ \begin{array}{cccccccccccc}
0 & 1 & 1 & 0 & 1 & 1 & 1 & 1 & 0 & 1 & 0 & 0 \\
1 & 1 & 0 & 1 & 1 & 1 & 1 & 0 & 1 & 0 & 0 & 0 \\
1 & 0 & 1 & 1 & 1 & 1 & 0 & 1 & 0 & 0 & 0 & 1 \\
0 & 1 & 1 & 1 & 1 & 0 & 1 & 0 & 0 & 0 & 1 & 1 \\
1 & 1 & 1 & 1 & 0 & 1 & 0 & 0 & 0 & 1 & 1 & 0 \\
1 & 1 & 1 & 0 & 1 & 0 & 0 & 0 & 1 & 1 & 0 & 1 \\
1 & 1 & 0 & 1 & 0 & 0 & 0 & 1 & 1 & 0 & 1 & 1 \\
1 & 0 & 1 & 0 & 0 & 0 & 1 & 1 & 0 & 1 & 1 & 1 \\
0 & 1 & 0 & 0 & 0 & 1 & 1 & 0 & 1 & 1 & 1 & 1 \\
1 & 0 & 0 & 0 & 1 & 1 & 0 & 1 & 1 & 1 & 1 & 0 \\
0 & 0 & 0 & 1 & 1 & 0 & 1 & 1 & 1 & 1 & 0 & 1 \\
0 & 0 & 1 & 1 & 0 & 1 & 1 & 1 & 1 & 0 & 1 & 0
\end{array} \right] \]
\caption{The $A$ submatrix of $U$.}
\label{fig:golayAmat}
\end{center}
\end{figure}
Due to the form of element and the listing chosen, the twelve by twelve identity matrix thus occupies the upper left-most and lower right-most blocks of the group ring matrix.
The matrix $A$ is reverse circulant and sits in the lower left-most and upper-right most blocks of the group ring matrix.
All reverse circulant matrices are symmetric, equal to their own transposes.
The fact that $A$ is symmetric leads directly to $U$ being symmetric.
Thus elements of the form $1+a\mathbf{f}$ in the group ring $\mathbb{Z}_2 \mathbf{D}_{24}$ under the listing $\{1,b,b^2,\ldots,b^{11},a,ab,ab^2,\ldots,ab^{11}\}$ have symmetric group ring matrices comprised of the twelve by twelve identity matrix and a twelve by twelve reverse circulant matrix.

The matrix $U$ is a matrix with entries in the field $\mathbb{Z}_2$.
Thus the row space of $U$ is a binary linear block code.
We will call this code $\mathcal{C}$.
Over the next few sections we will show that $\mathcal{C}$ is a $(24,12,8)$ linear block code and hence is the unique extended binary Golay code.
Along the way we will show that the code is self-dual, doubly even, quasi-$12$ cyclic and also that it is an ideal in the group ring.
We'll start by showing that the code is of dimension twelve.

\section{Dimension Twelve}
We prove here that the dimension of the code is twelve using a principle from linear algebra.
The principle states that the dimension of the row space of a matrix plus the dimension of its null space is equal to the number of columns that it has~\cite[p.~245]{mac99}.
The proof centres on the fact that the element $u$ when squared is equal to zero, a fact we now discuss.

\subsection{The Zero Divisor $u$}
\label{sect:golayzdu}
We have stated without proof that the element $u$ is a zero divisor.
Hence another element of the group ring must multiply with it to produce zero.
In fact it multiplies with itself to give zero; the element $u$ when squared is zero.
In the group ring it does not take long to multiply the terms of $u^2$ out in order to prove this.
The calculation is marginally hastened by the following argument: $u^2 = (1+a\mathbf{f})^2 = 1^2 + 2a\mathbf{f} + (a\mathbf{f})^2 = 1 + a^2(\mathbf{f}^{\textrm{T}}\mathbf{f}) = 1 + (\mathbf{f}^{\textrm{T}} \mathbf{f})$.
Thus $u$ squared is equal to zero if and only if $(\mathbf{f}^{\textrm{T}} \mathbf{f})$ is equal to one.
There is an even quicker way to verify that $u$ squared equals zero however, using its group ring matrix and a little algebra.

Thanks to the ring isomorphism between the group ring $\mathbb{Z}_2 \mathbf{D}_{24}$ and every ring of group ring matrices~\cite{hur07,hur09}, the product $u^2$ equals $0$ if and only if the group ring matrix $U$ when squared equals the zero matrix.
Working block-wise with the matrix $U$ we can see that:
\begin{equation*}
\begin{split}
&\qquad U^2 = \underline{0}_{24} \\
&\Leftrightarrow \left[ \begin{array}{cc} I & A \\ A & I \end{array} \right] \left[ \begin{array}{cc} I & A \\ A & I \end{array} \right] = \underline{0}_{24} \\
&\Leftrightarrow \left[ \begin{array}{cc} I + A^2 & A + A \\ A + A & I + A^2 \end{array} \right]  = \left[ \begin{array}{cc} \underline{0}_{12} & \underline{0}_{12} \\ \underline{0}_{12} & \underline{0}_{12} \end{array} \right] \\
&\Leftrightarrow \left[ \begin{array}{cc} I + A^2 & \underline{0}_{12} \\ \underline{0}_{12} & I + A^2 \end{array} \right]  = \left[ \begin{array}{cc} \underline{0}_{12} & \underline{0}_{12} \\ \underline{0}_{12} & \underline{0}_{12} \end{array} \right] \\
\end{split}
\end{equation*}
Here we have denoted the $m$ by $m$ zero matrix as $\underline{0}_{m}$ for $m$ a positive integer.
Thus the matrix $U$ when squared is equal to zero if and only if the matrix $A$ squared is equal to the identity matrix.

In multiplying $A$ by $A$ we take the dot products of each of the rows of $A$ with each of the columns of $A$, giving the entries of the product.
The matrix $A$ when squared is equal to the identity matrix if the dot product of row $i$ with column $j$ is one when $i$ is equal to $j$ and zero otherwise, for $i$ and $j$ from one to twelve.

Since $A$ is symmetric, its rows are equal to its columns in the same order.
Hence we only need show that the dot product of distinct rows of $A$ is zero and the dot product of each row with itself is one.
The latter is easy.
The first row is of odd weight and hence its dot product with itself is one.
All the other rows are cyclic shifts of the first row and so are of odd weight.
The dot product of each row with itself is thus one.

In order to show that the dot product of distinct rows of $A$ is zero we will use a property of $A$ that we will re-use at other times later in the chapter.
Thus we will give it its own section.

\subsection{Pairs of Rows of $A$}
\label{sect:pairsrows}
In this section we will prove that the dot product of each pair of distinct rows of $A$ is equal to the dot product of row one with one of the rows two to seven of $A$.
Consider two arbitrary rows of $A$: rows $i$ and $j$ where $1 \leq i < j \leq 12$.
All the rows of $A$ are cycles of each other since $A$ is reverse circulant.
Thus rows $i$ and $j$ are both cycles of the first row.

The dot product of two vectors is unaffected by cycling the two vectors by the same amount.
Hence we can cycle rows $i$ and $j$ by the same amount and their dot product will remain unchanged.
We can cycle row $i$ to become the first row and cycle row $j$ by the same amount.
The two cycled rows have the same dot product as the two rows before the cycling.
We can also cycle row $j$ until it becomes the first row and cycle row $i$ by the same amount, leaving their dot product unchanged.
Rows $i$ and $j$ have thus the same dot product as two potentially distinct pairs of rows where one of the rows in the pair is the first row of $A$.

The only case in which these two pairs of rows are not distinct is when the two rows are six cycles apart.
Six is half the number of rows in $A$.
Otherwise the two pairs of rows are distinct.
The first pair of rows is row one with row $j+(13-i)$ modulo twelve since it takes $13-i$ modulo twelve cycles to turn the $i^{\textrm{th}}$ row into the first.
The second is row one with row $i+(13-j)$ modulo twelve since row one and row $j$ are $(13-j)$ modulo twelve cycles apart.
Figure \ref{fig:rowcycles} illustrates this connection for rows five and ten of $A$.
\begin{figure}
\begin{center}
\begin{tabular}{|c|c|}
	\hline
	Row & Components \\
	\hline	
	5  & $1\ 1\ 1\ 1\ 0\ 1\ 0\ 0\ 0\ 1\ 1\ 0$ \\
	10 & $1\ 0\ 0\ 0\ 1\ 1\ 0\ 1\ 1\ 1\ 1\ 0$ \\
	\hline
\rowcolor[rgb]{0.8,0.8,0.8}	+  & $0\ 1\ 1\ 1\ 1\ 0\ 0\ 1\ 1\ 0\ 0\ 0$\\
	\hline
	\hline
	1  & $0\ 1\ 1\ 0\ 1\ 1\ 1\ 1\ 0\ 1\ 0\ 0$\\
	6  & $1\ 1\ 1\ 0\ 1\ 0\ 0\ 0\ 1\ 1\ 0\ 1$\\
	\hline
\rowcolor[rgb]{0.8,0.8,0.8}	+  & $1\ 0\ 0\ 0\ 0\ 1\ 1\ 1\ 1\ 0\ 0\ 1$\\
	\hline
	\hline
	1  & $0\ 1\ 1\ 0\ 1\ 1\ 1\ 1\ 0\ 1\ 0\ 0$\\
	8  & $1\ 0\ 1\ 0\ 0\ 0\ 1\ 1\ 0\ 1\ 1\ 1$\\
	\hline
\rowcolor[rgb]{0.8,0.8,0.8}	+  & $1\ 1\ 0\ 0\ 1\ 1\ 0\ 0\ 0\ 0\ 1\ 1$\\
	\hline 
	
	
\end{tabular}
\caption{Three Combinations of Equal Weight}
\label{fig:rowcycles}
\end{center}
\end{figure}

The dot product of row one and row $i+(13-j)$ is therefore equal to that of row one with row $j+(13-i)$.
This implies that the dot product of row one with row $k$ is equal to that of row one with row $14-k$ for $k$ from two to seven.
Letting $k=i+(13-j)$ we can manipulate the equality (modulo twelve) to show that $j+(13-i)=14-k$.
In other words the dot product of the first row with any of the rows further down the matrix than the seventh row is equal to the dot product of the first row with one of the rows further up the matrix than the seventh row.

So the dot product of row one with row two is equal to that of row one with row twelve, that of row one with row three is equal to that of row one with row eleven, and so on.
Therefore we can check that the dot product of every pair of distinct rows of $A$ is zero by checking that the dot product of row one with each of the rows two to seven is zero.
A quick investigation of the matrix $A$ above shows this to be true.
Thus the matrix $U^2$ and the group ring element $u^2$ are both equal to zero in their respective rings.

The dot product is not the only operation to which the above argument can be applied.
For instance, the weight of each two row combination of $A$ is also equal to the weight of some combination of row one and one of the rows two to seven.
This observation will be used in later sections.
Furthermore we will use the fact that row one paired with row seven is the only combination involving row one that is not necessarily equal in weight to any of the others.
Right now we will determine the rank of $U$ and hence the dimension of the code $\mathcal{C}$.

\subsection{The Rank of $U$}
\label{sect:golayrankofu}
We will show that the rank of $U$ is exactly twelve by first showing that is at least twelve and then showing that it is at most twelve.
The rank of a matrix is the dimension of its row space which is equal to the maximum number of rows that are linearly independent.
Every identity matrix is of full rank since all of its rows are linearly independent.
The twelve by twelve identity matrix forms the top left block of the matrix $U$.
Hence the rank of $U$ is at least twelve.

The fact that the matrix $U$ squared is equal to zero limits its rank.
From linear algebra the rank of a matrix plus the dimension of its null space is equal to the number of columns it has~\cite[p.~245]{mac99}.
The matrix $U$ has twenty-four columns and we've just seen that its rank is at least twelve.
All of the columns of $U$ are contained in its null space since squaring it gives zero.
The matrix is equal to its own transpose and so its columns are equal to its rows and therefore the rows of $U$ are contained in its null space.
Hence the null space of $U$ is of dimension at least twelve and its rank is therefore at most twenty-four minus twelve which is twelve.
Combined with the previous argument therefore the rank of $U$ must be exactly twelve.

Obviously then the code $\mathcal{C}$ which is the span of the rows of $U$ is a $(24,12)$ linear block code.
Moreover the first twelve rows of $U$ are linearly independent and so form a basis for the code.
A generator matrix for the code $\mathcal{C}$ is thus the matrix $G = [ I | A ]$.
This matrix is of an interesting form a fact discussed in the next section.

\subsection{Reverse Circulant Generator Matrices}
Having created the generator matrix for our code we will make a quick aside to note a property of the construction that has been of interest in the literature.
The generator matrix that has been constructed consists entirely of two square $k$ by $k$ submatrices, one on the left that is circulant (the identity matrix) and another on the right that is reverse circulant (the matrix $A$).
Code generator matrices of this form are called reverse circulant generator matrices~\cite[p.~377]{huf03}.
A closely related type of generator matrix that has received a good deal of coverage in the literature are those of the form $G' = [B|C]$ where $B$ and $C$ are  circulant matrices.
Generator matrices of that form are called double circulant generator matrices~\cite{gul98}.
Switching the $i^{\textrm{th}}$ column of the submatrix $A$ in our generator matrix with its $(13-i)^{\textrm{th}}$, for $i$ from one to six, will result in a double circulant matrix that generates the same code (up to equivalence).

Now we will return to the main points points of interest in our construction.
We have constructed a code $\mathcal{C}$ of length twenty-four and dimension twelve.
Should this code have minimum distance eight then it is the extended binary Golay code.
Before proving that this is in fact the minimum distance of $\mathcal{C}$ we will show that the code is self-dual and doubly even.
We start with the code's self-duality.


\section{Self-Duality}
As defined in section \ref{sect:dualcode} a code is self-dual if it is exactly its own dual code.
The code $\mathcal{C}$ in vector form is the row space of the group ring matrix $U$.
We saw in section \ref{sect:golayrankofu} that the rows of $U$ are contained in its dual code.
Hence the code $\mathcal{C}$ is self-orthogonal.

Of course we also saw also that the row space of $U$ is of dimension twelve and the null space of $U$ is of dimension twenty-four minus twelve, which is also twelve.
Since the code is a subspace, the row space and null space of $U$ are exactly the same.
Hence the code $\mathcal{C}$ is self-dual.

As stated in section \ref{sect:selfdualzdcodes} a zero divisor code generated by a zero divisor $x$ is self-dual if and only if $xx^{\textrm{T}} = 0$ and the rank of $x$ is half the order of the group~\cite{hur09}.
These two conditions hold true for $u$, since the group ring matrix $U$ is symmetric and of rank twelve.
Thus we can see that the code is self-dual from both the vector code perspective and the group ring perspective.
In the next section we use the self-duality of the code to observe that it is doubly even.

\section{Doubly Evenness}
A \emph{doubly even code} is a code in which the weight of every non-zero codeword is divisible by four.
Any binary and self-orthogonal linear block code generated by a matrix whose rows are all of weight divisible by four is doubly even~\cite[p.~10]{huf03}.
Since $\mathcal{C}$ is self-dual, it is self-orthogonal.
Also the first row of $G$ is of weight eight and all the other rows are permutations of this row.
Hence all of the rows are of weight eight.
Therefore $\mathcal{C}$ is a doubly even code.

The fact that the code is doubly even greatly aids the calculation of its minimum distance.
Another facet of the code that aids the calculation of its minimum distance is its quasi cyclic nature.

\section{Quasi Cyclicity}
Inspection of the matrix $U$ in section \ref{sect:golayrgmatrix} will convince the reader that the code $\mathcal{C}$ is in fact quasi-$12$ cyclic.
The first twelve rows of $U$ are of the form $G=[I|A]$ and the last twelve rows are of the form $G'=[A|I]$.
The first twelve rows generate the code.

For two vectors $\underline{x}_1$ and $\underline{x}_2$ we call the vector resulting from appendage of the components of $\underline{x}_2$ to the end of those in $\underline{x}_1$ the \emph{concatenation} $\underline{x}_1 \underline{x}_2$ of $\underline{x}_2$ to $\underline{x}_1$.
Let $\underline{c}$ be a combination of rows of $G$ and let $\underline{a}$ and $\underline{b}$ be the vectors consisting first and last twelve components respectively of $\underline{c}$, in the same order that the components appear in $\underline{c}$.
Then the same rows from $G'$ combine to give the concatenation $\underline{b}\underline{a}$ of the vector $\underline{a}$ to the vector $\underline{b}$.
The $1$'s in the first twelve components of a codeword indicate which combination of rows of $G$ is the codeword.

For example in figure \ref{fig:quasicyclic} we can see that the combination $\underline{x}=\underline{a}\underline{b}$ of rows one, five and ten of $G$ is equal to the twelfth right cyclic shift of the combination of rows one, five and ten of $G'$.
This latter combination is equal to the combination of rows four, six, seven, nine and ten of $G$.
\begin{figure}
\begin{center}
\begin{tabular}{|r|c|c|}
	\hline
	$G$ & $I$ & $A$ \\
	\hline
	 $1$ & $1\ 0\ 0\ 0\ 0\ 0\ 0\ 0\ 0\ 0\ 0\ 0$ & $0\ 1\ 1\ 0\ 1\ 1\ 1\ 1\ 0\ 1\ 0\ 0$ \\
	 $5$ & $0\ 0\ 0\ 0\ 1\ 0\ 0\ 0\ 0\ 0\ 0\ 0$ & $1\ 1\ 1\ 1\ 0\ 1\ 0\ 0\ 0\ 1\ 1\ 0$ \\
	$10$ & $0\ 0\ 0\ 0\ 0\ 0\ 0\ 0\ 0\ 0\ 1\ 0$ & $1\ 0\ 0\ 0\ 1\ 1\ 0\ 1\ 1\ 1\ 1\ 0$ \\								
	\hline
\rowcolor[rgb]{0.8,0.8,0.8} $+$ & $1\ 0\ 0\ 0\ 1\ 0\ 0\ 0\ 0\ 1\ 0\ 0$ & $0\ 0\ 0\ 1\ 0\ 1\ 1\ 0\ 1\ 1\ 0\ 0$ \\
	\hline
	\hline
	$G'$ & $A$ & $I$ \\
	\hline
	 $1$ & $0\ 1\ 1\ 0\ 1\ 1\ 1\ 1\ 0\ 1\ 0\ 0$ & $1\ 0\ 0\ 0\ 0\ 0\ 0\ 0\ 0\ 0\ 0\ 0$ \\
	 $5$ & $1\ 1\ 1\ 1\ 0\ 1\ 0\ 0\ 0\ 1\ 1\ 0$ & $0\ 0\ 0\ 0\ 1\ 0\ 0\ 0\ 0\ 0\ 0\ 0$ \\
	$10$ & $1\ 0\ 0\ 0\ 1\ 1\ 0\ 1\ 1\ 1\ 1\ 0$ & $0\ 0\ 0\ 0\ 0\ 0\ 0\ 0\ 0\ 0\ 1\ 0$ \\								
	\hline
	\rowcolor[rgb]{0.8,0.8,0.8} $+$ & $0\ 0\ 0\ 1\ 0\ 1\ 1\ 0\ 1\ 1\ 0\ 0$ & $1\ 0\ 0\ 0\ 1\ 0\ 0\ 0\ 0\ 1\ 0\ 0$ \\
	\hline
	\hline
	$G$ & $I$ & $A$ \\
	\hline
	 $4$ & $0\ 0\ 0\ 1\ 0\ 0\ 0\ 0\ 0\ 0\ 0\ 0$ & $0\ 1\ 1\ 1\ 1\ 0\ 1\ 0\ 0\ 0\ 1\ 1$ \\
	 $6$ & $0\ 0\ 0\ 0\ 0\ 1\ 0\ 0\ 0\ 0\ 0\ 0$ & $1\ 1\ 1\ 0\ 1\ 0\ 0\ 0\ 1\ 1\ 0\ 1$ \\
	 $7$ & $0\ 0\ 0\ 0\ 0\ 0\ 1\ 0\ 0\ 0\ 0\ 0$ & $1\ 1\ 0\ 1\ 0\ 0\ 0\ 1\ 1\ 0\ 1\ 1$ \\							
	 $9$ & $0\ 0\ 0\ 0\ 0\ 0\ 0\ 0\ 1\ 0\ 0\ 0$ & $0\ 1\ 0\ 0\ 0\ 1\ 1\ 0\ 1\ 1\ 1\ 1$ \\								
	$10$ & $0\ 0\ 0\ 0\ 0\ 0\ 0\ 0\ 0\ 1\ 0\ 0$ & $1\ 0\ 0\ 0\ 1\ 1\ 0\ 1\ 1\ 1\ 1\ 0$ \\						
	\hline
	\rowcolor[rgb]{0.8,0.8,0.8} $+$ & $0\ 0\ 0\ 1\ 0\ 1\ 1\ 0\ 1\ 1\ 0\ 0$ & $1\ 0\ 0\ 0\ 1\ 0\ 0\ 0\ 0\ 1\ 0\ 0$ \\
	\hline
\end{tabular}
\caption{The Quasi-12 Cyclic Nature of $\mathcal{C}$}
\label{fig:quasicyclic}
\end{center}
\end{figure}

The row space of $G'$ is contained in the row space of $U$, and so it is contained in the code $\mathcal{C}$.
Thus $\underline{b}\underline{a}$ is a codeword in $\mathcal{C}$ whenever $\underline{a}\underline{b}$ is.
Therefore the code $\mathcal{C}$ is quasi-$12$ cyclic, or in other words is quasi cyclic of index two.

\section{Minimum Distance}
\label{sect:golaymindistance}
As we know, the extended binary Golay code is the only linear block code of length twenty-four, dimension twelve and minimum distance eight~\cite[p.~401]{huf03}.
Our code $\mathcal{C}$ has so far been shown to be of length twenty-four and dimension twelve.
We will now see that $\mathcal{C}$ is of minimum distance eight and therefore it is the extended binary Golay code.
In the following arguments we will use many of the properties of the construction that we have already highlighted.
First we will consider how the matrix $A$ is related to the minimum distance of the code.

\subsection{The Matrix $A$ and the Code}
The generator matrix $G$ of the code $\mathcal{C}$ is of the form $G=[I|A]$.
Every codeword is a combination of rows of $G$ and every combination of rows of $G$ is a codeword.
Remember that the minimum distance of a linear block code is equal to the minimum of the weights of its codewords.
To show the minimum distance of the code $\mathcal{C}$ is eight, we will show that the weight of every non-zero combination of the rows of $G$ is of weight at least eight.

Now note that the combination of any $i$ rows of a $k$ by $k$ identity matrix is of weight exactly $i$ for $i$ from zero to $k$.
Like in the previous section, let $\underline{a}$ be the vector consisting of the first twelve components of a combination of $i$ rows of $G$ in order and let $\underline{b}$ be the last twelve in order where $i$ is an integer between zero and twelve inclusive.
The vector $\underline{a}$ is then a combination of $i$ rows of the identity matrix and so is of weight exactly $i$.
Thus the weight of the combination of the rows of $G$ is $i$ plus the weight of the combination of those rows from $A$.
That is, the weight of the vector $\underline{b}$.
This is illustrated with an example in figure \ref{fig:rowsimat}.
The weight of the combination of the three rows in the figure is equal to three from the first twelve components plus the weight from the last twelve.
\begin{figure}
\begin{center}
\begin{tabular}{|r|c|c|}
	\hline
	Row & $\underline{a}$ & \underline{b} \\
	\hline
	$1$ & $1\ 0\ 0\ 0\ 0\ 0\ 0\ 0\ 0\ 0\ 0\ 0$ & $0\ 1\ 1\ 0\ 1\ 1\ 1\ 1\ 0\ 1\ 0\ 0$ \\
	$5$ & $0\ 0\ 0\ 0\ 1\ 0\ 0\ 0\ 0\ 0\ 0\ 0$ & $1\ 1\ 1\ 1\ 0\ 1\ 0\ 0\ 0\ 1\ 1\ 0$ \\
	$10$ & $0\ 0\ 0\ 0\ 0\ 0\ 0\ 0\ 0\ 0\ 1\ 0$ & $1\ 0\ 0\ 0\ 1\ 1\ 0\ 1\ 1\ 1\ 1\ 0$ \\								
	\hline
\rowcolor[rgb]{0.8,0.8,0.8} $+$ & $1\ 0\ 0\ 0\ 1\ 0\ 0\ 0\ 0\ 1\ 0\ 0$ & $0\ 0\ 0\ 1\ 0\ 1\ 1\ 0\ 1\ 1\ 0\ 0$ \\
	\hline
	& weight 3 & weight 5 \\
	\hline
\end{tabular}
\caption{Rows of the Identity Matrix}
\label{fig:rowsimat}
\end{center}
\end{figure}

The weight of a combination of rows from $G$ is thus the weight of the combination of those rows from $A$ plus the number of rows involved in the combination.
In the next section we use this fact to show that many of the codewords in $\mathcal{C}$ cannot be of weight less than eight.

\subsection{Four Rows or More}
Using the relationship between the weights of the combinations of the rows of $G$ and those of $A$ we can see that there are a large number of combinations of rows of $G$ that are obviously of weight at least eight.
Every combination of $i$ rows of $G$ must be of weight at least $i$ for $i$ from zero to twelve.
Thus every combination of five or more rows of $G$ is of weight at least five.
Now, the code $\mathcal{C}$ is doubly even and so every combination of five or more rows of $G$ must in fact combine to give a codeword of at least weight eight.
Hence the only non-zero codewords of $\mathcal{C}$ that could possibly be of weight less than eight are those that are combinations of four or less rows of $G$.

A four row combination of $G$ being of weight four would imply that those four rows of $A$ are linearly dependent, combining to give the zero vector of length twelve.
Earlier in the matrix calculation in section \ref{sect:golayzdu}, showing that $U^2$ was equal to zero, we saw that $A^2$ was equal to the identity matrix.
This implies that $A$ is its own inverse and any matrix that has an inverse is of full rank~\cite[p.~246]{mac99}.
Hence the rows of $A$ are linearly independent over $\mathbb{Z}_2$ and so no non-zero combination of them is the zero vector.
Thus every non-zero combination of four rows of $A$ has non-zero weight.
Hence every four row combination of the rows of $G$ is of weight at least eight.
We therefore are only left to prove that no three-, two- or one-row combinations of $G$ have weight less than eight.
These cases are straightforward to prove.

\subsection{One and Two Row Combinations}
\label{sect:onetworowcombs}
The first row of $G$ is of weight eight and all other rows are permutations of this row.
Hence no single row is of weight less than eight.
Furthermore we saw in section \ref{sect:pairsrows} that the weight of each two row combination of $A$ is equal to the weight of one of the combinations of row one with rows two to seven.
The weight of these combinations is easy to check: rows two to six combine with row one to give combinations of weight six, and the row one and row seven combination is of weight ten.
Combining those weights with the weight of two rows from the identity matrix, we see that all of the two row combinations of $G$ are of weight eight or more.
Thus the only non-zero combinations of rows of $G$ that could lead to codewords of weight less than eight are those of three rows.
We will now prove that none of these three row combinations have weight less than eight.

\subsection{Three Row Combinations}
We use the quasi-$12$ cyclic property to show that no three row combination of $G$ is of weigh less than eight.
The code $\mathcal{C}$ is doubly even.
Thus the only possible weight that a non-zero combination of rows of $G$ can have that is less than eight is four.

Suppose $\underline{c}$ is a weight four combination of three rows of $G$.
Let $\underline{a}$ be the first twelve components of $\underline{c}$ and let $\underline{b}$ be the last twelve.
Then $\underline{a}$ is of weight three and $\underline{b}$ is of weight one.

Since $\underline{c} = \underline{a} \underline{b}$ is a codeword, then so too is the vector $\underline{c}' = \underline{b}\underline{a}$.
The generator matrix $G$ is in standard form and is of rank twelve so the first twelve components of a codeword in the code $\mathcal{C}$ indicate exactly the rows of $G$ that combine to generate that codeword.
When the $i^{\textrm{th}}$ component of such a codeword is $1$ for $i$ from one to twelve the $i^{\textrm{th}}$ row of $G$ is involved in the combination.
The number of non-zero components in these first twelve components is the number of rows of $G$ that combined to generate the codeword.
Therefore in the case of $\underline{c}'$ a single row of $G$ forms the codeword.
But all of the rows of $G$ are of weight eight, not weight four.
Thus we have a contradiction and our original supposition that three rows of $G$ combine to give a codeword of weight four must be incorrect.

So every three row combination of $G$ is at least of weight eight.
This concludes the proof that the minimum distance of $\mathcal{C}$ is at least eight.
Since the rows of $G$ are all of weight eight and are codewords in themselves the minimum distance of $G$ is exactly eight.
We will now discuss the fact that the code is actually an ideal in group ring.

\section{The Code as an Ideal}
We have used the full row space of the group ring matrix $U$ in constructing the code $\mathcal{C}$.
In section \ref{sect:vectorforms} we saw that this means that the code is the principal left ideal of $u$ in the group ring.
Thus the $(24,12,8)$ extended binary Golay code is the principal left ideal of the element $u = 1 + a(b + b^2 + b^4 + b^5 + b^6 + b^7 + b^9)$ in the group ring $\mathbb{Z}_2 \mathbf{D}_{24}$.

This concludes the first part of this chapter in which we show that $u$ generates the code.
In the second part of this chapter we will show that twenty-three other elements in the group ring also generate the code.
We then prove that these are the only elements of their form that generate the code.
We start by discussing the twenty-three zero divisors.

\section{The Other Generators}
\label{sect:othergens}
In the next few sections we will see first that the other twenty-three zero divisors listed in section \ref{sect:listgolayzds} also generate the $(24,12,8)$ extended binary Golay code.
We do this by showing that they generate equivalent codes to that generated by $u = 1 + a \mathbf{f} = 1 + a \mathbf{f} = a(b + b^2 + b^4 + b^5 + b^6 + b^7 + b^9)$, which we have just dealt with.
The other generators are $1+a b^i \mathbf{f}$ for $i$ from one to eleven and $1+a b^j \mathbf{f}^{\textrm{T}}$ for $j$ from zero to eleven.
We first deal with the former type in the next section, followed by the latter type in the section after that.

\subsection{Those without the Transpose}
\label{sect:withouttranspose}
We deal now with the generating elements of the form $1 + a b^i \mathbf{f}$ for $i$ from one to eleven.
The fact that $u^2 = (1+a \mathbf{f})^2$ is equal to zero implies that $u_i = (1+ a b^i \mathbf{f})^2$ is also equal to zero for all $i$.
Working through the multiplication:
\begin{align*}
(1+ a b^i \mathbf{f})^2 
&= 1 + 2 ( a b^i \mathbf{f} ) + ( a b^i \mathbf{f})^2 \\
&= 1 + 0 + ( a b^i \mathbf{f})^2 \\
&= 1 +  a b^i \mathbf{f} a b^i \mathbf{f} \\
&= 1 +  a b^i a \mathbf{f}^{\textrm{T}} b^i \mathbf{f} \\
&= 1 +  a b^i a b^i \mathbf{f}^{\textrm{T}} \mathbf{f} \\
&= 1 +  \mathbf{f}^{\textrm{T}} \mathbf{f}.
\end{align*}
In section \ref{sect:golayzdu} we proved that $\mathbf{f}^{\textrm{T}} \mathbf{f}$ is equal to $1$ and so $(1 + a b^i \mathbf{f})^2$ is zero for $i$ from one to eleven.
Thus the maximum rank of the group ring matrix $U_i$ of $u_i$ is twelve.

Notice that, just like in $u$'s case, the group ring matrix $U_i$ is of the form:
\[\left[ \begin{array}{c|c}
I & A_i \\
\hline
A_i & I
\end{array} \right] \]
where $I$ is the twelve by twelve identity matrix and $A_i$ is a twelve by twelve reverse circulant matrix.
Thus the rank of $U_i$ is exactly twelve and the first twelve rows of it generate its row space.
Here we are still using the listing $\{ 1 , b , b^2 , \ldots , b^{11}  , a , ab , ab^2 , \ldots , ab^{11} \}$ of $\mathbf{D}_{24}$.

The components of the first row of $A_i$ are the coefficients in $u_i$ of the elements $ab^j$ for $j$ from zero to eleven written in order.
The other rows of $A_i$ are determined by this first row since $A_i$ is reverse circulant.
Multiplying $\mathbf{f}$ by a power of $b$ simply cycles the first row of $A$.
This has the effect of cycling $A$'s columns.
Thus $A_i$ is the matrix resulting from cycling the columns of $A$ $i$ times.
Hence the row space of the group ring matrix $U_i$ is generated by a matrix $G_i = [I|A_i]$ that is a column permutation of the generator matrix $G = [I|A]$ of the group ring code $\mathcal{C}$ from section \ref{sect:golayrgmatrix}.

Thus the code that is the row space of the group ring matrix of any element $u_i = 1 + a b^i \mathbf{f}$ for $i$ from one to eleven is equivalent to the code $\mathcal{C}$.
Therefore it is the extended binary Golay code.
Notice that we can easily adapt the above proof to prove that any element of the form $1+ab^k \mathbf{h}$ generates an equivalent code to that of $1+a \mathbf{h}$ where $\mathbf{h}$ is an arbitrary sum of powers of $b$ for $k$ from one to eleven, so long as $\mathbf{h}^2$ is equal to one.
We will use such an adaptation of the proof in section \ref{sect:thefirstcolumn}.
We now move on to those elements of the form $1 + a b^j \mathbf{f}^{\textrm{T}}$ for $j$ from zero to eleven.

\subsection{Those with the Transpose}
In this section we turn our attention to the elements of the form $1+ a b^j \mathbf{f}^{\textrm{T}}$ for $j$ from zero to twelve.
Suppose $1 + a \mathbf{f}^{\textrm{T}}$ does generate the extended binary Golay code.
The argument in the previous section showing that $1 + a b^i \mathbf{f}$ generates an equivalent code to $1 + a \mathbf{f}$ will also show that elements of the form $1 + a b^j \mathbf{f}^{\textrm{T}}$ generate an equivalent code to $1 + a b^j \mathbf{f}^{\textrm{T}}$.
The only necessary change to the argument is to substitute $\mathbf{f}^{\textrm{T}}$ for every $\mathbf{f}$.
Thus we only need show that $1 + a \mathbf{f}^{\textrm{T}}$ generates an equivalent code to $u=1 + a \mathbf{f}$.

The fact that $u^2$ equals zero again implies that $(1 + a \mathbf{f}^{\textrm{T}})^2$ is also equal to zero:
\begin{align*}
(1+ a \mathbf{f}^{\textrm{T}})^2 
&= 1 + 2 ( a \mathbf{f}^{\textrm{T}} ) + ( a \mathbf{f}^{\textrm{T}})^2 \\
&= 1 + ( a \mathbf{f}^{\textrm{T}})^2 \\
&= 1 +  a \mathbf{f}^{\textrm{T}} a \mathbf{f}^{\textrm{T}} \\
&= 1 +  a a \mathbf{f} \mathbf{f}^{\textrm{T}} \\
&= 1 +  \mathbf{f} \mathbf{f}^{\textrm{T}} \\
&= 1 +  \mathbf{f}^{\textrm{T}} \mathbf{f}.
\end{align*}
Thus $1 + a \mathbf{f}^{\textrm{T}}$ generates a code of dimension twelve and the first twelve rows of its group ring matrix generate the code.

This generator matrix is of the form $[I|B]$ where $B$ is the twelve by twelve matrix on the right-hand side of figure \ref{fig:Aofftranspose}.
The matrix on the left-hand side of the figure is the matrix $A$ in the generator matrix $G=[I|A]$ of the code $\mathcal{C}$ generated by $u$.
They have been placed side by side for comparison.
\begin{figure}
\begin{center}
\begin{minipage}{0.5\linewidth}
\begin{center}
$ \left[ \begin{array}{c@{\hspace{5pt}}c@{\hspace{5pt}}c@{\hspace{5pt}}c@{\hspace{5pt}}c@{\hspace{5pt}}c@{\hspace{5pt}}c@{\hspace{5pt}}c@{\hspace{5pt}}c@{\hspace{5pt}}c@{\hspace{5pt}}c@{\hspace{5pt}}c}
0 & 1 & 1 & 0 & 1 & 1 & 1 & 1 & 0 & 1 & 0 & 0 \\
1 & 1 & 0 & 1 & 1 & 1 & 1 & 0 & 1 & 0 & 0 & 0 \\
1 & 0 & 1 & 1 & 1 & 1 & 0 & 1 & 0 & 0 & 0 & 1 \\
0 & 1 & 1 & 1 & 1 & 0 & 1 & 0 & 0 & 0 & 1 & 1 \\
1 & 1 & 1 & 1 & 0 & 1 & 0 & 0 & 0 & 1 & 1 & 0 \\
1 & 1 & 1 & 0 & 1 & 0 & 0 & 0 & 1 & 1 & 0 & 1 \\
1 & 1 & 0 & 1 & 0 & 0 & 0 & 1 & 1 & 0 & 1 & 1 \\
1 & 0 & 1 & 0 & 0 & 0 & 1 & 1 & 0 & 1 & 1 & 1 \\
0 & 1 & 0 & 0 & 0 & 1 & 1 & 0 & 1 & 1 & 1 & 1 \\
1 & 0 & 0 & 0 & 1 & 1 & 0 & 1 & 1 & 1 & 1 & 0 \\
0 & 0 & 0 & 1 & 1 & 0 & 1 & 1 & 1 & 1 & 0 & 1 \\
0 & 0 & 1 & 1 & 0 & 1 & 1 & 1 & 1 & 0 & 1 & 0
\end{array} \right] $
\end{center}
\end{minipage}\begin{minipage}{0.5\linewidth}
\begin{center}
$ \left[ \begin{array}{c@{\hspace{5pt}}c@{\hspace{5pt}}c@{\hspace{5pt}}c@{\hspace{5pt}}c@{\hspace{5pt}}c@{\hspace{5pt}}c@{\hspace{5pt}}c@{\hspace{5pt}}c@{\hspace{5pt}}c@{\hspace{5pt}}c@{\hspace{5pt}}c}
0 & 0 & 0 & 1 & 0 & 1 & 1 & 1 & 1 & 0 & 1 & 1 \\
0 & 0 & 1 & 0 & 1 & 1 & 1 & 1 & 0 & 1 & 1 & 0 \\
0 & 1 & 0 & 1 & 1 & 1 & 1 & 0 & 1 & 1 & 0 & 0 \\
1 & 0 & 1 & 1 & 1 & 1 & 0 & 1 & 1 & 0 & 0 & 0 \\
0 & 1 & 1 & 1 & 1 & 0 & 1 & 1 & 0 & 0 & 0 & 1 \\
1 & 1 & 1 & 1 & 0 & 1 & 1 & 0 & 0 & 0 & 1 & 0 \\
1 & 1 & 1 & 0 & 1 & 1 & 0 & 0 & 0 & 1 & 0 & 1 \\
1 & 1 & 0 & 1 & 1 & 0 & 0 & 0 & 1 & 0 & 1 & 1 \\
1 & 0 & 1 & 1 & 0 & 0 & 0 & 1 & 0 & 1 & 1 & 1 \\
0 & 1 & 1 & 0 & 0 & 0 & 1 & 0 & 1 & 1 & 1 & 1 \\
1 & 1 & 0 & 0 & 0 & 1 & 0 & 1 & 1 & 1 & 1 & 0 \\
1 & 0 & 0 & 0 & 1 & 0 & 1 & 1 & 1 & 1 & 0 & 1 
\end{array} \right] $
\end{center}
\end{minipage}
\caption{The Matrix $A$ on the left and the matrix $B$ on the right.}
\label{fig:Aofftranspose}
\end{center}
\end{figure}

In the following we will show that the codes generated by $[I|A]$ and $[I|B]$ are equivalent.
We do this by showing that $A$ can be constructed from $B$ by row and column permutations.
Any matrix formed by a permutation of the rows of $G$ generates the same code.
A matrix formed by a column permutation of $A$ will, on being appended to the identity matrix, generate an equivalent code to $\mathcal{C}$.
Thus the result will show that $1 + a \mathbf{f}^{\textrm{T}}$ generates the extended binary Golay code.

We start with the matrix $B$.
The first permutation we apply is a row permutation where the first and last, the second and second-last, the third and third-last, and so on, rows are interchanged.
The result is the matrix on the left-hand side of figure \ref{fig:BtoAmatrices}.
The second permutation we apply is the similar column permutation where the first and last, the second and second-last, the third and third-last, and so on, columns are interchanged.
The resulting matrix is that on the right-hand side of the figure.
The final permutation is a column permutation in which the columns are cycled two steps to the right.
The result is the matrix $A$ on the left-hand side of the previous figure \ref{fig:Aofftranspose}.
\begin{figure}
\begin{center}
\begin{minipage}{0.5\linewidth}
\begin{center}
$ 	\left[ \begin{array}{c@{\hspace{5pt}}c@{\hspace{5pt}}c@{\hspace{5pt}}c@{\hspace{5pt}}c@{\hspace{5pt}}c@{\hspace{5pt}}c@{\hspace{5pt}}c@{\hspace{5pt}}c@{\hspace{5pt}}c@{\hspace{5pt}}c@{\hspace{5pt}}c}
1 & 0 & 0 & 0 & 1 & 0 & 1 & 1 & 1 & 1 & 0 & 1 \\
1 & 1 & 0 & 0 & 0 & 1 & 0 & 1 & 1 & 1 & 1 & 0 \\
0 & 1 & 1 & 0 & 0 & 0 & 1 & 0 & 1 & 1 & 1 & 1 \\
1 & 0 & 1 & 1 & 0 & 0 & 0 & 1 & 0 & 1 & 1 & 1 \\
1 & 1 & 0 & 1 & 1 & 0 & 0 & 0 & 1 & 0 & 1 & 1 \\
1 & 1 & 1 & 0 & 1 & 1 & 0 & 0 & 0 & 1 & 0 & 1 \\
1 & 1 & 1 & 1 & 0 & 1 & 1 & 0 & 0 & 0 & 1 & 0 \\
0 & 1 & 1 & 1 & 1 & 0 & 1 & 1 & 0 & 0 & 0 & 1 \\
1 & 0 & 1 & 1 & 1 & 1 & 0 & 1 & 1 & 0 & 0 & 0 \\
0 & 1 & 0 & 1 & 1 & 1 & 1 & 0 & 1 & 1 & 0 & 0 \\
0 & 0 & 1 & 0 & 1 & 1 & 1 & 1 & 0 & 1 & 1 & 0 \\
0 & 0 & 0 & 1 & 0 & 1 & 1 & 1 & 1 & 0 & 1 & 1
\end{array} \right] $
\end{center}
\end{minipage}\begin{minipage}{0.5\linewidth}
\begin{center}
$ \left[ 	\begin{array}{c@{\hspace{5pt}}c@{\hspace{5pt}}c@{\hspace{5pt}}c@{\hspace{5pt}}c@{\hspace{5pt}}c@{\hspace{5pt}}c@{\hspace{5pt}}c@{\hspace{5pt}}c@{\hspace{5pt}}c@{\hspace{5pt}}c@{\hspace{5pt}}c}
1 & 0 & 1 & 1 & 1 & 1 & 0 & 1 & 0 & 0 & 0 & 1 \\
0 & 1 & 1 & 1 & 1 & 0 & 1 & 0 & 0 & 0 & 1 & 1 \\
1 & 1 & 1 & 1 & 0 & 1 & 0 & 0 & 0 & 1 & 1 & 0 \\
1 & 1 & 1 & 0 & 1 & 0 & 0 & 0 & 1 & 1 & 0 & 1 \\
1 & 1 & 0 & 1 & 0 & 0 & 0 & 1 & 1 & 0 & 1 & 1 \\
1 & 0 & 1 & 0 & 0 & 0 & 1 & 1 & 0 & 1 & 1 & 1 \\
0 & 1 & 0 & 0 & 0 & 1 & 1 & 0 & 1 & 1 & 1 & 1 \\
1 & 0 & 0 & 0 & 1 & 1 & 0 & 1 & 1 & 1 & 1 & 0 \\
0 & 0 & 0 & 1 & 1 & 0 & 1 & 1 & 1 & 1 & 0 & 1 \\
0 & 0 & 1 & 1 & 0 & 1 & 1 & 1 & 1 & 0 & 1 & 0 \\
0 & 1 & 1 & 0 & 1 & 1 & 1 & 1 & 0 & 1 & 0 & 0 \\
1 & 1 & 0 & 1 & 1 & 1 & 1 & 0 & 1 & 0 & 0 & 0
\end{array} \right] $
\end{center}
\end{minipage}
\caption{Column and row permutations of $B$ showing code equivalence.}
\label{fig:BtoAmatrices}
\end{center}
\end{figure}

The matrix $B$ is thus a combination of row and column permutations of the matrix $A$.
The group ring element $1 + a \mathbf{f}^{\textrm{T}}$ therefore generates the extended binary Golay code.
As a consequence so do the elements $1 + a b^j \mathbf{f}^{\textrm{T}}$ for $j$ from one to eleven.

This concludes the proof the elements $1+a b^i \mathbf{f}$ for $i$ from one to eleven and $1+a b^j \mathbf{f}^{\textrm{T}}$ for $j$ from zero to eleven generate the extended binary Golay code along with $u$.
We will now discuss why these twenty-four group ring elements are capable of generating the code, while the others of the form $1+a \mathbf{f}'$ are not.

\section{The Only Zero Divisors}
\label{sect:setdifferences}
The computer search we conducted in our initial analysis of the extend binary Golay code as a group ring code found only the twenty-four zero divisors in the last section to generate the code.
The computer only searched for zero divisors of the form $1+a \mathbf{h}$ where $\mathbf{h}$ is a sum of powers of $b$.
Having algebraically proven that these elements did in fact generate the code, we turned our attention to the investigation of why they generate the code.
The hope was that this would offer insight allowing the construction of longer type II codes.
The result was the following proof that the twenty-four elements from the last section are the only ones of the form $1+a \mathbf{h}$ that do generate the code.
In the next section we highlight two properties that together are unique to the twenty-four elements.

\subsection{Combinations of Two Rows}
Consider the matrix $A$ in figure \ref{fig:golayAmat}.
In section \ref{sect:onetworowcombs} we listed the weights of the combinations of row one and each of rows two to seven of $A$.
The combinations of row one with each of rows two to six are of weight six.
This implies that the combinations of row one with each of rows eight to twelve are also of weight six, as was shown in section \ref{sect:pairsrows}.
The combination of rows one and seven is of weight ten.
In total for the eleven two-row combinations of $A$ involving row one, ten are of weight six and one is of weight ten.

Each row of $A$ is of weight seven.
In combining two rows we lose one from each of the two weights of seven every time both rows contain a $1$ in the same component.
We say that two rows `share' a $1$-component whenever they both contain a $1$ in that component.
This is illustrated in the example in figure \ref{fig:sharedcomponents}.
Rows one and seven of $A$ share $1$-components in components two and eight.
We say the $1$ from one row `cancels' the $1$ in the same component in their combination since the sum of two $1$'s is $0$.
This terminology helps in the following argument that $\mathbf{h}$ in the generator $1+ a \mathbf{h}$ must be of weight seven to lead to the generation of the extended binary Golay code.
\begin{figure}
\begin{center}
\begin{tabular}{|r|cccccccccccc|}
	\hline
	Components & $1$ & $ 2$ & $ 3$ & $ 4$ & $ 5$ & $ 6$ & $ 7$ & $ 8$ & $ 9$ & $10$ & $11$ & $12$ \\
	\hline
	Row $1$ & $0$ & $1$ & $1$ & $0$ & $1$ & $1$ & $1$ & $1$ & $0$ & $1$ & $0$ & $0$ \\
	Row $7$ & $1$ & $1$ & $0$ & $1$ & $0$ & $0$ & $0$ & $1$ & $1$ & $0$ & $1$ & $1$ \\
	\hline
\end{tabular}
\caption{Rows one and seven share components two and eight.}
\label{fig:sharedcomponents}
\end{center}
\end{figure}

\subsubsection{Only Weight Seven}
\label{sect:onlywtseven}
Let $\mathbf{h}$ be a sum of powers of the generator $b$ of $\mathbf{D}_{24}$.
The extended binary Golay code is doubly even.
The rows of any generator matrix of the code must thus have weights divisible by four.
For an element of the form $1+ a \mathbf{h}$ to generate the code, $\mathbf{h}$ must thus be of weight congruent to three modulo four.
The possible weights of $\mathbf{h}$ are thus three, seven and eleven.
There are only twelve group elements involved in $\mathbf{h}$ and thus it cannot have any higher weight.

The minimum distance of the extended binary Golay code is eight.
The minimum distance of any code generated by an element in which $\mathbf{h}$ is of weight three is at most four.
The generator of the code is a codeword itself and it has weight four.
Were $\mathbf{h}$ to have weight eleven then the combination of row one and two of the group ring matrix of $1+ a \mathbf{h}$ would have weight four.
So again the minimum distance of the code would be at most four.

The only possible weight of $\mathbf{h}$ is therefore seven.
When $\mathbf{h}$ is of weight seven, the group ring element $1+a\mathbf{h}$ is of weight eight.
All of the twenty-four generators in section \ref{sect:othergens} are of weight eight.
This property along with another elucidated in the next section are together unique in the group ring to the twenty-four zero divisors listed in section \ref{sect:listgolayzds}.

\subsubsection{Shared $1$-Components}
\label{sect:sharedonecomps}
Since the combinations of row one of $A$ with each of the rows two to six and eight to twelve are of weight six, those rows each share four $1$-components with row one.
Row seven shares only two with row one.
This is the second property that makes the twenty-four weight eight zero divisors unique in the group ring.
They all have ten of the other rows sharing four $1$-components with row one and one row sharing two with row one.

We saw in section \ref{sect:golayzdu} that $u^2$ is only equal to zero if the rows of $A$ are orthogonal.
They are only orthogonal if each pair of rows of $A$ shares an even number of $1$-components.
This argument can be extended to any element of the form $1+ a \mathbf{h}$ in $\mathbb{Z}_2 \mathbf{D}_{24}$.
Such an element, according to the listing $\mathbf{D}_{24} = \{ 1 , b, \ldots , b^{11} , a , ab , \ldots , ab^{11} \}$ has a group ring matrix of the form:
\[\left[ \begin{array}{c|c}
I & B \\
\hline
B & I
\end{array} \right] \]
where $I$ and $B$ are twelve by twelve matrices.
The matrix $B$ is reverse circulant, just like in the case of $u$ in section \ref{sect:golayrgmatrix}.
The element is only a zero divisor with itself when each of $B$'s rows shares an even number of components with row one of $B$.

Thus row one of such a $B$ must share either zero, two, four, six, eight or ten $1$-components with each of the other rows.
Two rows of $B$ must share at least one $1$-component as $B$ has only twelve columns and each row is of weight seven.
Two rows of weight seven can share a maximum of seven components.
Should two rows share six $1$'s then the weight of the combination of those two rows would be four.
Thus the code that is the row space of the group ring matrix of $1 + a \mathbf{h}$ is only of minimum distance eight if each of the rows two to twelve of $B$ shares either two or four $1$-components with the first row.

The total number of $1$-components the other rows of $B$ share with row one is forty-two.
This is exactly the product of seven, the weight of $\mathbf{h}$, and six.
The correspondence is due to $B$'s reverse circulant nature.
Each of the $1$'s in the first row cancels each of the other $1$'s in the row exactly once as we move down through the rows of the matrix.

Thus the only way for eleven rows to provide forty two cancellations is if one of them provides two cancellations and the other provide four each.
The one that provides two must be the seventh row.
If one of the rows two to six provided only two then so would one of the rows eight to twelve.
Since only one row can provide exactly two, it must be row seven.
So only generators of the form $1+ a \mathbf{h}$ where $\mathbf{h}$ is of weight seven and all of the rows two to six of $B$ share four $1$-components and row seven shares two, can generate the extended binary Golay code.
We will see in a moment that only the twenty-four zero divisors we have listed in section \ref{sect:othergens} have these properties.
First we will discuss the shared $1$-components from the group ring perspective.

\section{The Multiset of Differences}
\label{sect:multidiffs}
In this section we look from the group ring perspective at the $1$-components shared between the first row of $B$ and the other rows.
The correspondence between these sharings and the elements of $b$ with non-zero coefficient in $\mathbf{h}$ is interesting.
We show that their `multiset of differences', as defined below, has an interesting structure.
The notation used here is the same as in the last few sections---we are looking at the situations in which an element $1+a \mathbf{h}$ can generate the extended binary Golay code.
The $B$ matrix is as defined in section \ref{sect:sharedonecomps}: it is the reverse circulant submatrix in the top-right of the group ring matrix of $1+ a \mathbf{h}$.
As always in this chapter, we are using the listing $\{ 1 , b, \ldots , b^{11} , a , ab , \ldots , ab^{11} \}$ of $\mathbf{D}_{24}$.

Row $i+1$ of the group ring matrix is the vector form of the group ring codeword $g_{i+1} (1 + a \mathbf{h})$ where $g_{i+1}$ is the $(i+1)^{\textrm{th}}$ element of the listing.
This was discussed in section \ref{sect:vectorforms}.
Thus row $i+1$ of the group ring matrix corresponds to the group ring element $b^iu = b^i( 1 + a \mathbf{h} ) = b^i + a (b^{-i} \mathbf{h})$ for $i$ an integer from zero to eleven.
Row $i+1$ of the matrix $B$ has a $1$ in the $(j+1)^{\textrm{th}}$ component when $b^{-i}b^k = b^j$ where $b^k$ has a non-zero coefficient in $\mathbf{h}$.
The $1$'s in rows two to twelve are in the same components that row one has $1$'s when $b^j = b^l$ where $b^l$ is some power of $b$ in $\mathbf{h}$ with non-zero coefficient.
As we saw in the last section, $\mathbf{h}$ must be of weight seven in order for $1 + a \mathbf{h}$ to generate the extended binary Golay code.
Thus rows two to twelve of $B$ have a total of eleven times seven $1$'s, forty-two of which are shared with row one's $1$-components.
Row one and row $(i+1)$ therefore share a $1$-component whenever $b^{-i}b^k = b^l$, or $b^i = b^{k-l}$, where $b^k$ and $b^l$ are powers of $b$ with non-zero coefficients in $\mathbf{h}$.

The integer $k$ never equals $l$ as the same $1$-component passes through all the of the columns while moving down through the matrix $B$ and thus never cancels with itself.
Therefore the set $\{ b^{k-l} \mid b^k , b^l \in \mathbf{f} , k \neq l \}$, where by ``$\in$'' we mean `has non-zero coefficient in', is a set of powers of $b$ that describes exactly when the $1$'s from row one cancel with the $1$'s from the other rows.
We will call this set \emph{the multiset of differences} and denote it $\mathcal{D}$.
The number of times this set contains $b^i$ is the number of $1$-components that the $(i+1)^{\textrm{th}}$ row cancels from the first row.

As an example of a multiset of differences, we'll calculate that pertaining to the extended binary Golay code generator element $\mathbf{f}$ in $u = 1 + a \mathbf{f}$ from section \ref{sect:listgolayzds}.
The element $\mathbf{f}$ is thus $\mathbf{f} = b + b^2 + b^4 + b^5 + b^6 + b^7 + b^9$.
The multiset of differences is given in figure \ref{fig:differencesoff}.
\begin{figure}
\begin{center}
$
\begin{array}{rl}
\begin{array}{r} \mathcal{D} = \{   \\ \qquad \\ \qquad \\ \qquad \\ \qquad \\ \qquad \\ \qquad \\ \qquad \\ \qquad \end{array} &
\begin{array}{@{ }c@{ }c@{ }c@{ }c@{ }c@{ }c@{ }c@{ }c@{ }}
	\begin{array}{c} b^{1-2}, \\ b^{2-1}, \\ b^{4-1}, \\ b^{5-1}, \\ b^{6-1}, \\ b^{7-1}, \\ b^{9-1}, \\ \qquad  \\ \qquad \end{array} &
	\begin{array}{c} b^{1-4}, \\ b^{2-4}, \\ b^{4-2}, \\ b^{5-2}, \\ b^{6-2}, \\ b^{7-2}, \\ b^{9-2}, \\ \qquad  \\ \qquad \end{array} &
	\begin{array}{c} b^{1-5}, \\ b^{2-5}, \\ b^{4-5}, \\ b^{5-4}, \\ b^{6-4}, \\ b^{7-4}, \\ b^{9-4}, \\ \qquad  \\ \qquad \end{array} &
	\begin{array}{c} b^{1-6}, \\ b^{2-6}, \\ b^{4-6}, \\ b^{5-6}, \\ b^{6-5}, \\ b^{7-5}, \\ b^{9-5}, \\ \qquad  \\ \qquad \end{array} &
	\begin{array}{c} b^{1-7}, \\ b^{2-7}, \\ b^{4-7}, \\ b^{5-7}, \\ b^{6-7}, \\ b^{7-6}, \\ b^{9-6}, \\ \qquad  \\ \qquad \end{array} &
	\begin{array}{c} b^{1-9}, \\ b^{2-9}, \\ b^{4-9}, \\ b^{5-9}, \\ b^{6-9}, \\ b^{7-9}, \\ b^{9-7}  \\ \qquad  \\ \qquad \end{array} &
	\begin{array}{l} \qquad \\ \qquad \\ \qquad \\ \qquad \\ \qquad \\ \qquad \\ \} \\ \qquad  \\ \qquad  \end{array}	
\end{array} \\
\begin{array}{r} \Rightarrow \mathcal{D} = \{   \\ \qquad \\ \qquad \\ \qquad \\ \qquad \\ \qquad \\ \qquad \\ \qquad \\ \qquad \end{array} &
\begin{array}{@{ }c@{ }c@{ }c@{ }c@{ }c@{ }c@{ }c@{ }c@{ }}
	\begin{array}{c} b^{11}, \\ b^{1}, \\ b^{3}, \\ b^{4}, \\ b^{5}, \\ b^{6}, \\ b^{8},  \\ \qquad  \\ \qquad \end{array} &
	\begin{array}{c} b^{9}, \\ b^{10}, \\ b^{2}, \\ b^{3}, \\ b^{4}, \\ b^{5}, \\ b^{7},  \\ \qquad  \\ \qquad \end{array} &
	\begin{array}{c} b^{8}, \\ b^{9}, \\ b^{11}, \\ b^{1}, \\ b^{2}, \\ b^{3}, \\ b^{5},  \\ \qquad  \\ \qquad \end{array} &
	\begin{array}{c} b^{7}, \\ b^{8}, \\ b^{10}, \\ b^{11}, \\ b^{1}, \\ b^{2}, \\ b^{4},  \\ \qquad  \\ \qquad \end{array} &
	\begin{array}{c} b^{6}, \\ b^{7}, \\ b^{9}, \\ b^{10}, \\ b^{11}, \\ b^{1}, \\ b^{3},  \\ \qquad  \\ \qquad \end{array} &
	\begin{array}{c} b^{4}, \\ b^{5}, \\ b^{7}, \\ b^{8}, \\ b^{9}, \\ b^{10}, \\ b^{2}  \\ \qquad  \\ \qquad  \end{array} &
	\begin{array}{l} \qquad \\ \qquad \\ \qquad \\ \qquad \\ \qquad \\ \qquad \\ \}  \\ \qquad  \\ \qquad \end{array}	
\end{array} \\
\begin{array}{r} \Rightarrow \mathcal{D} = \{   \\ \qquad \\ \qquad \\ \qquad \\ \qquad \\ \qquad \\ \qquad \\ \qquad \\ \qquad \\ \qquad \\ \qquad \end{array} &
\begin{array}{@{ }c@{ }c@{ }c@{ }c@{ }c@{ }c@{ }}
	\begin{array}{c} b^{1}, \\ b^{2}, \\ b^{3}, \\ b^{4}, \\ b^{5}, \\ b^{6}, \\ b^{7}, \\ b^{8}, \\ b^{9}, \\ b^{10}, \\ b^{11}, \end{array} &
	\begin{array}{c} b^{1}, \\ b^{2}, \\ b^{3}, \\ b^{4}, \\ b^{5}, \\ b^{6}, \\ b^{7}, \\ b^{8}, \\ b^{9}, \\ b^{10}, \\ b^{11}, \end{array} &
	\begin{array}{c} b^{1}, \\ b^{2}, \\ b^{3}, \\ b^{4}, \\ b^{5}, \\ \qquad \\ b^{7}, \\ b^{8}, \\ b^{9}, \\ b^{10}, \\ b^{11}, \end{array} &
	\begin{array}{c} b^{1}, \\ b^{2}, \\ b^{3}, \\ b^{4}, \\ b^{5}, \\ \qquad \\ b^{7}, \\ b^{8}, \\ b^{9}, \\ b^{10}, \\ b^{11} \end{array} &
	\begin{array}{l} \qquad \\ \qquad \\ \qquad \\ \qquad \\ \qquad \\ \qquad \\ \qquad \\ \qquad \\ \qquad \\ \qquad \\ \} \end{array}	
\end{array}
\end{array}
$
\caption{The multiset of differences of $\mathbf{f}$.}
\label{fig:differencesoff}
\end{center}
\end{figure}
These calculations support our earlier assertion that rows two to six and rows eight to twelve share four $1$-components in common with row one and row seven shares two with it.
We say the multiset of differences $\mathcal{D}$ has the form $(\mathbf{10} \times 4) + (\mathbf{1} \times 2)$ meaning ten elements appear in it four times and one element appears twice.

The element $b^{l-k}$ is a member of $\mathcal{D}$ whenever $b^{k-l}$ is.
The elements $b^{k-l}$ and $b^{l-k}$ are inverses of each other and so the appearance of an element implies its inverse.
Suppose we calculate the elements of the set $\mathcal{D}'  = \{ b^{k-l} \mid b^k , b^l \in \mathbf{f} , k > l \}$ where $k$ is greater than $l$.
Then we can calculate the number of times each element appears in $\mathcal{D}$ by simply adding the number of times it appears in $\mathcal{D}'$ to the number of times its inverse appears in $\mathcal{D}'$.

In the remainder of this chapter we will show that the only elements of the form $1 + a \mathbf{h}$ that can generate the extended binary Golay code are those in which $\mathbf{h}$ is of weight seven and has a multiset of differences of the form $(\mathbf{10} \times 4) + (\mathbf{1} \times 2)$.

\section{Possible Sets of Differences}
In section \ref{sect:onlywtseven} we proved that a zero divisor of the form $1 + a \mathbf{h}$ where $\mathbf{h}$ is a sum of powers of $b$ can generate the extended binary Golay code only if $\mathbf{h}$ is of weight seven.
Furthermore in section \ref{sect:sharedonecomps} we showed that row seven in the $B$ submatrix of the group ring matrix of $1 + a \mathbf{h}$ must share two $1$-components with row one, and each of the other rows must share four.
We now prove that the twenty-four zero divisors listed in section \ref{sect:othergens} are the only ones of the form $1+ a \mathbf{h}$ with these properties.
The zero divisors listed there are $u = 1 + a \mathbf{f} = a(b + b^2 + b^4 + b^5 + b^6 + b^7 + b^9)$, $1+a b^i \mathbf{f}$ for $i$ an integer from one to eleven and $1+a b^j \mathbf{f}^{\textrm{T}}$ for an integer $j$ from zero to eleven.
We start by discussing the combination of rows one, five and nine of such a $B$.
This combination is interesting because row one is four cycles from row five, which is four cycles from row nine, which is four cycles from row one.
The matrix $B$ only contains twelve rows.

\subsubsection{The Combination of Rows One, Five and Nine}
The combination of rows one, five and nine of $B$ in the group ring matrix of $1 + a \mathbf{h}$ has an interesting property.
The rows of the combination are evenly spaced throughout $B$.
Cycling the three rows four times means the first becomes the fifth row, the fifth row becomes the ninth and the ninth becomes the first.
The first four components of the combination are thus equal to the second four, which are equal to the last four in the same order.
This is illustrated in figure \ref{fig:rowsonefivenine}, where the $i^{\textrm{th}}$ component of row one of $B$ is labelled by $x_i$.
\begin{figure}
\begin{center}
$\begin{array}{|r||c|c|c|c||c|c|c|c||c|c|c|c|}
	\hline
	\textrm{Row One:}  & x_{ 1} & x_{ 2} & x_{ 3} & x_{ 4} & x_{ 5} & x_{ 6} & x_{ 7} & x_{ 8} & x_{ 9} & x_{10} & x_{11} & x_{12} \\
	\hline
	\textrm{Row Five:} & x_{ 5} & x_{ 6} & x_{ 7} & x_{ 8} & x_{ 9} & x_{10} & x_{11} & x_{12} & x_{ 1} & x_{ 2} & x_{ 3} & x_{ 4} \\
	\hline
	\textrm{Row Nine:} & x_{ 9} & x_{10} & x_{11} & x_{12} & x_{ 1} & x_{ 2} & x_{ 3} & x_{ 4} & x_{ 5} & x_{ 6} & x_{ 7} & x_{ 8} \\
	\hline
\end{array}$
\caption{The combination of rows one, five and nine.}
\label{fig:rowsonefivenine}
\end{center}
\end{figure}
This implies that the weight of the combination is a multiple of three.
The possible weights of the combination are thus three, six, nine and twelve.

The combination must also be of weight congruent to one modulo four.
This is due to the fact that the extended binary Golay code is doubly even and thus every combination of $i$ rows of $B$ must be of weight congruent to $(4-i)$ modulo four in order to facilitate the code's generation.
The only possibility for the weight of the combination of rows one, five and nine of $B$ is thus nine.
The first four components of the combination must then be a subvector of weight three, consisting of three $1$-components and a single $0$-component.
In the next section we show that exactly one of the $1$-components must arise from the combination of three $1$-components in the three-row combination.


\subsubsection{The First Four Components}
\label{sect:firstfourcomps}
The first four components in the combination of rows one, five and nine of $B$ are the column sums of the array in figure \ref{fig:firstfourcomps}.
\begin{figure}
\begin{center}
$\begin{array}{|c|c|c|c|}
	\hline
	 x_{ 1} & x_{ 2} & x_{ 3} & x_{ 4} \\
	\hline
	 x_{ 5} & x_{ 6} & x_{ 7} & x_{ 8} \\
	\hline
	 x_{ 9} & x_{10} & x_{11} & x_{12} \\
	\hline
\end{array}$
\caption{The array of the first four components.}
\label{fig:firstfourcomps}
\end{center}
\end{figure}
We've seen that three of these column sums are $1$ and one is $0$.
We will now show that this arises from a single combination of three $1$-components, two combinations of a single $1$-component with two $0$-components and a single combination of two $1$-components and a $0$-component.
Note that the entries in the array are the components of the first row of $B$ in order moving across the rows and from top to bottom.

Every column of three $1$-components in the array leads to the sharing of three $1$-components between row one and row five of $B$.
Suppose the first column of the array is such a column.
Examining figure \ref{fig:rowsonefivenine}, in which the first two rows are rows one and five of $B$, convinces us that the first, fifth and ninth components of those two rows will then both contain $1$-components.
This is true since those components of the combination are the sum of two of the components in the first column of the array in figure \ref{fig:firstfourcomps}.
Were column two of the array to contain all $1$-components then components two, six and ten of those two rows would both contain $1$-components, and so on.
In section \ref{sect:sharedonecomps} we proved that rows one and five must share exactly four $1$-components.
Thus there is at most one column of the array in figure \ref{fig:firstfourcomps} that contains all $1$-components.

Now we will show that there must be at least one column of the array containing all $1$-components.
Three of the column sums of the array must be $1$-components.
Seven of the twelve components in the first row of $B$ must be $1$-components in the first row of $B$.
These twelve components make up the entries of the array.
If there is no all $1$-component column then three of the columns would have to contain exactly two $1$-components and the other would have to contain exactly one.
This would mean there were three $0$ column sums and a single $1$ column sum.
Thus the array contains at least one, and so exactly one, column of all $1$-components.
We can assume without loss of generality that this is the first column of the array, a fact we discuss in the following section.


\subsubsection{The First Column}
\label{sect:thefirstcolumn}
In the previous section we showed that exactly one of the columns in the array in figure \ref{fig:firstfourcomps} contains all $1$-components.
In section \ref{sect:withouttranspose} we proved that the elements of the form $1 + a b^i \mathbf{h}$ for $i$ an integer from one to eleven generate the same code as the element $1 + a \mathbf{h}$ when ${\mathbf{h}}^{\textrm{T}} \mathbf{h}$ equals one.
Here we are assuming that $(1 + a \mathbf{h})^2$ is zero and so ${\mathbf{h}}^{\textrm{T}} \mathbf{h}$ must be one.

The proof relied on the fact that the multiplication of $\mathbf{h}$ by $b^i$ has the effect of cycling the first row and hence the columns of $B$.
Cycling the first row of $B$ has the effect of cycling the columns in the array in figure \ref{fig:firstfourcomps}.
Note that as a column is wrapped from left to right the elements within get permuted, but that will not affect the following arguments.
Thus we can assume without loss of generality that the column of the array containing all $1$-components is the first column.

In section \ref{sect:firstfourcomps} we showed that the other three columns of the array are one that contains exactly two $1$-components and two that contain exactly one $1$-component.
We will now see that, when the first column of the array contains all $1$-components, the column containing two $1$-components cannot be the third column of the array.

\subsubsection{The Third Column}
The first column of the array in figure \ref{fig:firstfourcomps} contains all $1$-components.
Every $1$-component in the third column thus contributes two $1$-component cancellations in the combination of rows one and seven of $B$.
This is evident in figure \ref{fig:rowsoneseven}.
\begin{figure}
\begin{center}
$\begin{array}{|r||c|c|c|c|c|c||c|c|c|c|c|c|}
	\hline
	\textrm{Row One:}   & x_{ 1} & x_{ 2} & x_{ 3} & x_{ 4} & x_{ 5} & x_{ 6} & x_{ 7} & x_{ 8} & x_{ 9} & x_{10} & x_{11} & x_{12} \\
	\hline
	\textrm{Row Seven:} & x_{ 7} & x_{ 8} & x_{ 9} & x_{10} & x_{11} & x_{12} & x_{ 1} & x_{ 2} & x_{ 3} & x_{ 4} & x_{ 5} & x_{ 6} \\
	\hline
\end{array}$
\caption{The combination of rows one and seven.}
\label{fig:rowsoneseven}
\end{center}
\end{figure}
The $1$-components in the first column of the array are in components $x_1$, $x_5$  and $x_9$.
When $x_7$ is a $1$-component then, in figure \ref{fig:rowsoneseven}, it cancels in columns one and seven the $1$-component denoted $x_1$.
When $x_3$ is, it cancels in columns three and nine the $1$-component denoted $x_9$ .
When $x_{11}$ is, it cancels in columns five and eleven of the $1$-component denoted $x_5$.

Thus every $1$-component in column three contributes two to the number of $1$-components shared between rows one and seven of $B$.
In section \ref{sect:sharedonecomps} we showed that rows one and seven must share exactly two $1$-components in order for $B$ to lead to the generation of the extended binary Golay code.
The third column of the array can therefore harbour only one $1$-component.
Since the column is either a one or a two $1$-component containing column, it must contain exactly one $1$-component.
Of course we have not stipulated which of the three components in the third column is that $1$-component.
We will work through each of the three possibilities in a moment.
First we discuss the second and fourth columns of the array.

\subsubsection{The Second and Fourth Columns}
One of columns two and four of the array in figure \ref{fig:firstfourcomps} contains two $1$-components and the other contains one.
In a moment we will see that in both cases we will find an extended binary Golay code generating element.
First we discuss the fact that the positioning of the $1$-component in the column containing one $1$-component determines the positioning of the $0$-component in the two $1$-component containing column.

Whether the one $1$-component containing column is the second or the fourth column of the array we can assume, again without loss of generality, that the single $1$-component in that column is the top-most component in the column.
Cycling the elements of the first row of $B$ four times will permute the rows of the array so that the first row becomes the second, the second becomes the third and the third becomes the first.
The column sums remain unaffected.
The cycling will also permute the components within the columns of $B$, but as discussed in section \ref{sect:thefirstcolumn}, the matrix after cycling will lead to the generation of the code if and only if the matrix did before the cycling.
Once the position of the $1$-component in that column has been decided, there is only one positioning of the two $1$-components in the other column that can occur.

Some of the components in the second and fourth columns can lead to shared $1$-components between the first and seventh rows of $B$, just as in the situation with the first and third columns of the array.
We have already reached our maximum number of such shared components and thus we need to ensure that no such sharings arise from the second and fourth columns.
Considering figure \ref{fig:rowsoneseven} we can identify the components that lead to such sharings.

The components in the second column of the array are $x_2$, $x_6$  and $x_{10}$ and those in column four are $x_4$, $x_8$ and $x_{12}$.
First we take the case in which the one $1$-component column is the second.
The component $x_2$ of the array is then the $1$-component.
In that case if the component $x_8$ is a $1$-component then rows one and seven share two extra $1$-components in columns two and eight in figure \ref{fig:rowsoneseven}.
Thus $x_8$ must be a $0$-component and $x_4$ and $x_{12}$ must both be $1$-components.

Second we take the case in which the column containing a single $1$-component is the fourth.
The component $x_4$ is then the $1$-component in that column.
In that case if the component $x_{10}$ is a $1$-component then rows one and seven share two extra $1$-components in columns four and ten of figure \ref{fig:rowsoneseven}.
Thus $x_{10}$ must be a $0$-component and $x_2$ and $x_6$ must both be $1$-components.

We are then only left with six possibilities to analyse, three from each of the two cases just discussed.
The three possibilities in each case are: that in which the $1$-component in column three of the array is $x_3$, that in which the component is $x_7$ and that in which it is $x_{11}$.
We take each of the two cases in turn, going quickly through the three possibilities in each.

\subsubsection{The First Case}
In the first case the array has the form displayed in figure \ref{fig:arraycaseone}.
\begin{figure}
\begin{center}
\[
\begin{array}{|c|c|c|c|}
\hline
1 & 1 & x_{ 3} & 1 \\
\hline
1 & 0 & x_{ 7} & 0 \\
\hline
1 & 0 & x_{11} & 1 \\
\hline
\end{array}
\]
\caption{The array in the first case.}
\label{fig:arraycaseone}
\end{center}
\end{figure}
One of the components $x_3$, $x_7$ and $x_{11}$ is a $1$-component.
When it is $x_3$ the first row of $B$ is the first row displayed to the right of the double vertical lines in figure \ref{fig:rowcaseone}.
In that situation rows one and two (which is the first cycle of row one) would share five $1$-components.
They are only allowed share exactly four.

When the $1$-component is $x_7$ then the first row of $B$ is the second row in figure \ref{fig:rowcaseone}.
In that situation rows one and two of $B$ would share only three $1$-components.
When $x_{11}$ is the $1$-component the first row of $B$ is then the third row in figure \ref{fig:rowcaseone}.
This is the first row of the matrix $B$ corresponding to that of the group ring element $1 + a b^5 \mathbf{f}^{\textrm{T}}$, one of generators listed in section \ref{sect:othergens}.
Thus only in the situation that $x_{11}$ is the $1$-component will $B$ lead to the generation of the extended binary Golay code, and that situation corresponds to on of the twenty-four zero divisors.
We now move on to the second case.
\begin{figure}
\begin{center}
$\begin{array}{|r||c|c|c|c|c|c|c|c|c|c|c|c|}
	\hline
	x_{ 3} = 1 & 1 & 1 & 1 & 1 & 1 & 0 & 0 & 0 & 1 & 0 & 0 & 1 \\
	\hline
	\hline
	x_{ 7} = 1 & 1 & 1 & 0 & 1 & 1 & 0 & 1 & 0 & 1 & 0 & 0 & 1 \\
	\hline
	\hline
	x_{11} = 1 & 1 & 1 & 0 & 1 & 1 & 0 & 0 & 0 & 1 & 0 & 1 & 1 \\
	\hline
\end{array}$
\caption{The first rows of $B$ in the first case.}
\label{fig:rowcaseone}
\end{center}
\end{figure}


\subsubsection{The Second Case}
In the second case the array has the form given in figure \ref{fig:arraycasetwo}.
\begin{figure}
\begin{center}
\[
\begin{array}{|c|c|c|c|}
\hline
1 & 1 & x_{ 3} & 1 \\
\hline
1 & 1 & x_{ 7} & 0 \\
\hline
1 & 0 & x_{11} & 0 \\
\hline
\end{array}
\]
\caption{The array in the second case.}
\label{fig:arraycasetwo}
\end{center}
\end{figure}
Again in this case, exactly one of $x_3$, $x_7$ and $x_{11}$ is a $1$-component.
When it is $x_3$ the first row of $B$ is the first row displayed to the right of the double vertical lines in figure \ref{fig:rowcasetwo}.
In that situation rows one and two (which is the first cycle of row one) would share five $1$-components.
Again, they are only allowed share exactly four.

When the $1$-component is $x_{11}$ then the first row of $B$ is the third row in figure \ref{fig:rowcasetwo}.
In that situation rows one and two of $B$ would share only three $1$-components.
When $x_7$ is the $1$-component the first row of $B$ is then the second row in figure \ref{fig:rowcasetwo}.
This is the first row of the matrix $B$ corresponding to that of the group ring element $1 + a b^{11} \mathbf{f}$, one of generators listed in section \ref{sect:othergens}.
Thus in case two, only in the situation that $x_7$ is the $1$-component will $B$ lead to the generation of the extended binary Golay code, and that situation corresponds to on of the twenty-four zero divisors
\begin{figure}
\begin{center}
$\begin{array}{|r||c|c|c|c|c|c|c|c|c|c|c|c|}
	\hline
	x_{ 3} = 1 & 1 & 1 & 1 & 1 & 1 & 1 & 0 & 0 & 1 & 0 & 0 & 0 \\
	\hline
	\hline
	x_{ 7} = 1 & 1 & 1 & 0 & 1 & 1 & 1 & 0 & 0 & 1 & 0 & 1 & 0 \\
	\hline
	\hline
	x_{11} = 1 & 1 & 1 & 0 & 1 & 1 & 1 & 1 & 0 & 1 & 0 & 0 & 0 \\
	\hline
\end{array}$
\caption{The first rows of $B$ in the second case.}
\label{fig:rowcasetwo}
\end{center}
\end{figure}

Interestingly (but perhaps unsurprisingly), case one and case two as above are connected.
Consider again the array in figure \ref{fig:firstfourcomps}.
These are the coefficients of the group ring element $a \mathbf{h}$ written according to the last twelve elements of the listing $\{1,b,\ldots,b^{11},a,ab,\ldots,ab^{11}\}$ of $\mathbf{D}_{24}$.
Transposing the group ring element $\mathbf{h}$ has the effect of switching the coefficients of the elements so that they become associated with the inverse of their corresponding group elements.

The component $x_i$ of the array is the coefficient of $b^{i-1}$ in $\mathbf{h}$ for $i$ from one to twelve.
Hence after transposing, the component $x_i$ is in the former place of $x_{(12-(i-1)+1)}$, or $x_{14-i}$ where the subscript indices are calculated modulo twelve. 
Thus the array becomes that in figure \ref{fig:transposedarray}.
Columns one and three of the array have stayed put, albeit with their entries permuted.
Columns two and four have become interchanged however, again with their entries permuted.
The distinction between case one and two was the placement of the column with two $1$-components.
In case one it was column four and in case two it was column two.
Thus the two cases are somewhat related.
\begin{figure}
\begin{center}
$\begin{array}{|c|c|c|c|}
	\hline
	 x_{ 1} & x_{12} & x_{11} & x_{10} \\
	\hline
	 x_{ 9} & x_{ 8} & x_{ 7} & x_{ 6} \\
	\hline
	 x_{ 5} & x_{ 4} & x_{ 3} & x_{ 2} \\
	\hline
\end{array}$
\caption{The transposed array.}
\label{fig:transposedarray}
\end{center}
\end{figure}
This concludes the proof that the zero divisors listed in section \ref{sect:othergens} are the only extended binary Golay code generators of their form in $\mathbb{Z}_2 \mathbf{D}_{24}$, drawing us to the conclusion of the chapter.

\section*{Conclusion}
In this chapter we have seen that the extended binary Golay code can be constructed as the principal ideal of twenty-four different zero divisors in the group ring $\mathbb{Z}_2 \mathbf{D}_{24}$.
The zero divisors can be used to construct generator matrices that generate quasi cyclic forms of the code.
Furthermore we explored some techniques that will aid us in the analysis of longer codes that are like the extended binary Golay code.
In the process it was proven that the twenty-four zero divisors given are the only ones of their specific form that generate the code in the group ring.

In the next chapter we discuss a code that is twice as long, the $(48,24,12)$ type II extremal code.
We use many similar methods to those used in this chapter in its construction.
In particular we exploit a very similar technique to that in the last few sections where we analyse the combination of three rows spaced evenly throughout a submatrix of the group ring matrix.
We use this to prove that the code constructed is in fact the $(48,24,12)$ type II code.