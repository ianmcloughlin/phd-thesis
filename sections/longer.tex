%!TEX root = ../thesis.tex
\chapter{Further Type II Codes}
\label{chap:longer}
In this chapter we will discuss some general properties of codes constructed from zero divisors in dihedral group rings.
Along the way we outline some ideas for future work in the area of dihedral group ring codes.
Towards the end of the chapter we give some zero divisors we have found that generate two codes that are the best known type II codes for their length: a $(72,36,12)$ code and a $(96,48,16)$ code.

We begin with section \ref{sect:generaldihedralcodes} where we discuss the general construction of type II codes from zero divisors in dihedral group rings.
As discussed in the previous chapters type II codes are binary, self-dual and doubly even codes~\cite[p.~96]{han98a}.
Such codes are called extremal if their minimum distance achieves the upper bound $4 \lfloor n / 24 \rfloor + 4$ where $n$ is the code's length~\cite[p.~270]{han98a}.
It is straight-forward to construct type II codes using the group rings.
Proofs are provided in sections \ref{sect:genselfdual}, \ref{sect:gendoublyeven}, \ref{sect:genquasicyclic} and \ref{sect:mindisteight} showing that, given the right kind of zero divisor, the code it generates is automatically self-dual, doubly even, generated by a reverse circulant generator matrix and quasi cyclic of index two.
Given a few further conditions on the zero divisor we can show algebraically that such codes are of minimum distance at least eight.

Illustrating these proofs, we give examples of zero divisors generating extremal type II codes of each length that is a multiple of eight, up to length forty-eight.
Amongst them are two codes we have constructed in previous chapters.
The $(8,4,4)$ extended Hamming code was constructed in chapter \ref{chap:intro} using a zero divisor in the group ring $\mathbb{Z}_2 \mathbf{D}_{8}$ consisting of the finite field with two elements $\mathbb{Z}_2$ and the dihedral group with eight elements $\mathbf{D}_{8}$.
This was followed by the construction of the $(24,12,8)$ extended binary Golay code in chapter \ref{chap:extgolaycode}.
It was constructed from a zero divisor in the group ring $\mathbb{Z}_2 \mathbf{D}_{24}$ where $\mathbf{D}_{24}$ is the dihedral group with twenty-four elements.
The proofs given here are more general forms of those given in those chapters.

For lengths forty-eight and above type II codes are extremal only if their minimum distances are at least twelve.
This is the case for the $(48,24,12)$ type II code constructed in another dihedral group ring, $\mathbb{Z}_2 \mathbf{D}_{48}$, in chapter \ref{chap:fortyeight}.
As we witnessed there, it is difficult to show algebraically that such codes are of minimum distances greater than eight.
We can however employ some algebraic techniques to hasten the calculation of their minimum distances.
These are discussed in section \ref{sect:genmindistchecking}.
They allow the quick analysis of the longer codes by computer.

Extremal type II codes of longer lengths than forty-eight have received much attention in the past.
The existence of such codes for lengths seventy-two and ninety-six, amongst others, is unknown.
Neil J. Sloane called for attention to be given to the case of length seventy-two in 1973~\cite{slo73}.
Furthermore, S. T. Dougherty and Masaaki Harada offered monetary prizes as recently as 2001 for the proof of such a code's existence or proof of its non-existence~\cite{kim08}.
The problem is still open to date.

In section \ref{sect:gen723612} we will give an example of a zero divisor we have found in the group ring $\mathbb{Z}_2 \mathbf{D}_{72}$ that generates a $(72,36,12)$ type II code.
While this code is not extremal, twelve is the greatest minimum distance for which a type II code of length seventy-two is known.
Indeed should a $(72,36,16)$ code not exist then this is the greatest minimum distance for a type II code of length seventy-two.
We were unable to find, using computer search, a zero divisor that generates a type II length seventy-two code of minimum distance sixteen.

Along with the extremal $(72,36,16)$ code, the existence of an extremal $(96,48,20)$ type II is a long-standing open problem.
Again we were unable to find a zero divisor to generate such a code.
However, a zero divisor generating a length ninety-six type II code of minimum distance sixteen was found.
This is discussed in section \ref{sect:gen964816}.
Sixteen is the highest minimum distance for which a length ninety-six type II code is known.

While we have almost exclusively dealt with type II codes in this thesis, it is quite easy to construct type I codes using the same techniques.
Such codes are discussed in section \ref{sect:typeIcodes}.
In section \ref{sect:genproperties} we discuss some reasons why the generators we have listed throughout this thesis generate the codes they do.
The goal of further research into this topic could be to construct longer type II codes with good minimum distances.
We conclude the chapter with a brief discussion of groups other than the dihedral ones, over which we have found generators for some of the same codes as we have already constructed.
There appears to be a connection between these groups and the automorphism groups of the codes.

Throughout the chapter we refer to possibilities for further research in the area of group ring codes.
We begin now with a general discussion of dihedral codes.

\section{Dihedral Codes}
\label{sect:generaldihedralcodes}
In the following chapter we will discuss some properties of codes generated by zero divisors in the group ring formed from the finite field with two elements and a general dihedral group with $2k$ elements.
We denote the dihedral group by $\mathbf{D}_{2k} = \langle a , b \mid a^2 , b^k , (ab)^2 \rangle$.
So $a$ is a generator of order two and $b$ is a generator of order $k$.
The zero divisors in question are of the form $u = 1 + a \mathbf{f}$ where $\mathbf{f}$ is a sum of powers of $b$, and where $u^2$ is zero.
The codes we consider are the principal left ideals of such $u$'s: $\mathcal{C} = (\mathbb{Z}_2 \mathbf{D}_{2k})u$.

The length of a group ring code is the order of the underlying group.
We will only consider type II codes.
Codes generated by zero divisors of the form $1 + a \mathbf{f}$ are not always type II but we show in the following sections when they are and when they are not.
Type II codes only exist at lengths a multiple of eight~\cite{huf05}.
Thus in the following $k$ we assume $k$ is even.
We now show that the codes in question are always of dimension $k$.

\section{Dimension $k$}
\label{sect:gendimension}
Let $\mathcal{C}$ be the code $(\mathbb{Z}_2 \mathbf{D}_{2k})u$, generated by a zero divisor of the form $u = 1 + a \mathbf{f}$ where $f$ is a sum of powers of the group element $b$ and where $u^2$ equals zero.
Under the listing $D_{2k} = \{1,b,b^2,\ldots,b^{k-1},a,ab,ab^2,\ldots,ab^{k-1}\}$ the group ring matrix $U$ of $u$ has the form:
\[\left[ \begin{array}{c|c}
I & A \\
\hline
A & I
\end{array} \right] \]
where $I$ is the $k$ by $k$ identity matrix and $A$ is a $k$ by $k$ reverse circulant matrix.
The row space of $U$ is a vector form of the code $\mathcal{C}$, as discussed in section \ref{sect:vectorforms}.

We use a well known result from linear algebra to show that the dimension of (this vector form of) $\mathcal{C}$ is $k$.
This result is that the rank of a matrix plus the dimension of its null space is equal to the number of columns it has~\cite[p.~245]{mac99}.
It is used here to show that rank of $U$ is less than or equal to the dimension of its null space.
We first prove that the rows of $U$ are contained in its own null space 

The element $u$ is equal to its own transpose.
The transpose $u^{\textrm{T}}$ of a group ring element $u$ is the element in which the coefficient in $u$ of each group element $g$ is the coefficient of $g^{-1}$.
Every element of the form $ab^i$ in $\mathbf{D}_n$ for $i$ from zero to $k = n/2$ is its own inverse.
All of the elements in $a \mathbf{f}$ are of this form.
The group identity element $1$ is also its own inverse.
Therefore the coefficient in $u$ of each element $g$ is that of $g^{-1} = g$ in the transpose $u^{\textrm{T}}$ of $u$.
This implies that $U$ is equal to its own transpose and thus the rows of $U$ are exactly its columns (albeit transposed).
Since $U^2$ is zero the dot product of each row with each column is zero and hence $U$'s columns are contained in its null space.
Hence all of $U$'s rows are contained in its null space.

The rank of $U$ is thus less than or equal to the dimension of its null space.
The rank plus the dimension of the null space is $2k$.
Therefore the rank of $U$ is at most $k$ and the dimension of the null space is $2k$ minus the rank.
Now notice that the rank of $U$ is at least $k$ since the $k$ by $k$ identity matrix forms its top left block.
Hence the rank and the dimension of the null space of $U$ are both $k$.
The dimension of $\mathcal{C}$, which is equal to the rank of $U$, is therefore $k$.
We can use some of the above facts to easily see that the code $\mathcal{C}$ is in fact self-dual.

\section{Self-Duality}
\label{sect:genselfdual}
In the last section we saw that the row space of $U$ was contained in its null space.
The row space of $U$ is a vector form of the code $\mathcal{C}$ and thus the null space of $U$ is the dual code of the code $\mathcal{C}$.
Recall that the dual code of a code is the set of all vectors orthogonal to each of the codewords.
Codes that are subsets of their dual codes are called self-orthogonal codes.
Thus $\mathcal{C}$ is self-orthogonal.

The rank of $U$ and the dimension of the null space of $U$ are both $k$.
Since one is contained in the other the row space and null space of $U$ must in fact be the same space.
Thus not only is the code $\mathcal{C}$ contained in its dual code, it is in fact exactly its dual code.
Such codes are called self-dual codes and thus $\mathcal{C}$ is a self-dual code.

Every self-dual code has dimension half its length~\cite[p.~6]{huf03}, in accordance with our calculation of $\mathcal{C}$'s dimension.
Self-dual codes also have the property that every non-zero codeword is of even weight~\cite[p.~338]{huf03}.
Some of these codes are doubly even, meaning that all of their non-zero codewords only have weights divisible by four.
We now discuss doubly even codes as they arise in the context of our group ring codes.

\section{Doubly Evenness}
\label{sect:gendoublyeven}
A doubly even code is a code for which every non-zero codeword has weight divisible by four.
Type II codes are by definition doubly even.
The generator $u$ of the code $\mathcal{C}$ is a codeword itself.
Thus for $\mathcal{C}$ to be doubly even $u$ itself must be of weight divisible by four.
Since $u$ is of the form $1 + a\mathbf{f}$, this is the case if and only if $\mathbf{f}$ is of weight congruent to three modulo four.

We can easily prove that when $u$ is of weight divisible by four the self-dual code it generates is doubly even.
First observe that the rows of $U$ are permutations of the first row, which is the vector form of $u$.
Thus when $u$ is of weight divisible by four all of the rows of $U$ are.
Self-orthogonal codes with generator matrices whose rows are all of weight divisible by four are doubly even~\cite[p.~10]{huf03}.
We saw in section \ref{sect:genselfdual} that $\mathcal{C}$ is self-orthogonal.

Since the code $\mathcal{C}$ in vector form is the row space of the matrix $U$, some subset of the rows of $U$ forms a basis for the code.
These rows can be used to form a generator matrix for $\mathcal{C}$.
Thus the self-orthogonal code $\mathcal{C}$ can be generated by a matrix all of the rows of which are of weight divisible by four.
The code $\mathcal{C}$ is therefore doubly even when $u$ is of weight divisible by four.

Group ring codes that are the principal left ideals of zero divisors of $u$'s form where $u$ is of weight divisible by four are thus type II codes.
Note that the code $\mathcal{C}$ is a binary self-dual code irrespective of whether of not $u$ is of weight divisible by four.
When $u$ is not of weight divisible by four the code is singly-even but not doubly even.
Binary self-dual codes that are singly even but not doubly even are called type I codes.
While in this thesis we are mainly interested in those of type II, we now take a brief digression to discuss those arising in the same way that are type I.

\section{Type I Codes}
\label{sect:typeIcodes}
Singly even codes are codes whose non-zero codewords each have weight divisible by two.
Self-dual binary codes that are singly even but not doubly even are called type I codes~\cite[p.~339]{huf03}.
In section \ref{sect:genselfdual} we showed that a code $\mathcal{C} = (\mathbb{Z}_2 \mathbf{D}_{2k})u$, where $u$ is a zero divisor of the form $1 + a \mathbf{f}$ and $u^2 = 0$, is self-dual.
All self-dual codes are singly even and some are doubly even~\cite[p.~338]{huf03}.
Thus when a binary self-dual code is not type II it is type I.

Dihedral codes as we have constructed them are doubly even whenever $u$ is of weight divisible by four.
The self-dual code $\mathcal{C}$ can not be doubly even when $u$ is not of weight divisible by four.
Thus the code must be singly even in that case.
So $\mathcal{C}$ is a type I code when the weight of $u$ is divisible by two but not by four.

Note the implication here that $u = 1 + a \mathbf{f}$ can only equal zero when squared if the weight of $u$ is even.
We've seen that such a $u$ generates a self-dual code when it multiplies with itself to produce zero.
The element $u$ is a codeword itself and every codeword in a self-dual code is singly even.
Thus $u$ must be of even weight.
This fact is also obvious from a purely theoretical group ring perspective.
The element $u$ when squared is zero if and only if $\mathbf{f}^{\textrm{T}} \mathbf{f}$ is one.
This is only possible if $\mathbf{f}$ is of odd weight since this is the only case in which the number of terms in the multiplication is odd.

As an example of a type I code we construct the unique~\cite{huf05} type I code of length eight.
The principal left ideal of the zero divisor $1 + ab^3$ in the group ring $\mathbb{Z}_2 \mathbf{D}_{8}$ is the code in question.
The zero divisor is easily seen to multiply with itself to produce zero.
It is of the form $1 + a \mathbf{f}$ and so generates a self-dual code.
The minimum distance of this code is two which, while being very small, is the best distance a type I code of length eight can achieve~\cite{huf05}.

More interestingly, an extremal Type I code of length sixteen is the principal left ideal of the zero divisor $u = 1 + a(1 + b + b^2 + b^4 + b^6)$ in the group ring $\mathbb{Z}_2 \mathbf{D}_{16}$.
Again the element $u$ when squared is zero.
Since $u$ is of weight six the code can not be doubly-even.
A vector form of the code can be constructed as the row space of any group ring matrix of $u$.
Such a matrix will only have rows of weight six and thus the code is singly even~\cite[p.~11]{huf03}.
The code is of minimum distance four which can be shown by simply calculating the weights of the combinations of the first row of $U$ with each of the rows two to five.
This fact will be proven in section \ref{sect:mindisteight}.
The proof relies on the fact that the code $\mathcal{C}$ is quasi cyclic of index two, a property we discuss now.

\section{Quasi Cyclicity}
\label{sect:genquasicyclic}
Again we are considering the case of a zero divisor $u = 1 + a \mathbf{f}$ the group ring $\mathbb{Z}_2 \mathbf{D}_{2k}$, where $u^2$ is zero.
The code $\mathcal{C}$ is the principal left ideal of this element in the group ring and is of dimension $k$.
We showed in section \ref{sect:gendimension} that under the listing $D_{2k} = \{1,b,b^2,\ldots,b^{k-1},a,ab,ab^2,\ldots,ab^{k-1}\}$ the group ring matrix $U$ of the element $u$ has the form:
\[\left[ \begin{array}{c|c}
I & A \\
\hline
A & I
\end{array} \right] \]
where $I$ is the $k$ by $k$ identity matrix and $A$ is a $k$ by $k$ reverse circulant matrix.
The row space of the group ring matrix is a vector form of the group ring code $\mathcal{C}$, as discussed in section \ref{sect:vectorforms}.
Since the code is of dimension $k$ and the first $k$ rows of $U$ are obviously linearly independent, the matrix $G = [I|A]$ is a generator matrix for the code in vector form.

Let $\underline{c}$ be a combination of rows of $G$ and let $\underline{a}$ be the subvector of $\underline{c}$ containing its first $k$ components and $\underline{b}$ that containing its last $k$.
Then the concatenation $\underline{d} = \underline{b} \underline{a}$ of the subvector $\underline{a}$ to the end of $\underline{b}$ must also be in $\mathcal{C}$.
This is due to the fact that every combination of the rows of $U$ is a codeword in $\mathcal{C}$.
The submatrix consisting of the last $k$ rows of $U$ is of the form $[A|I]$.
The vector $\underline{d}$ is the combination of the same rows of $[A|I]$ as are combined from $G$ to form the codeword $\underline{c}$.

Codes of length $2k$ for which the $k^{\textrm{th}}$ cyclic shift of each codeword is also a codeword in this way are called quasi cyclic codes of index two.
Thus every code generated by an element $u = 1 + a \mathbf{f}$ where $u^2$ equals zero can be generated quasi cyclically of index two by the generator matrix consisting of the first $k$ rows of $U$.
Note that we have made no reference in this discussion as to whether $\mathcal{C}$ is singly or doubly even.
Thus both type I and type II codes constructed in this way are quasi cyclic of index two.
This quasi cyclic property is very useful in assessing the minimum distance of such codes.
We discuss this topic now in relation to type II codes.

\section{Minimum Distance at Least Eight}
\label{sect:mindisteight}
In the following sections we will assume that the zero divisor $u = 1 + a \mathbf{f}$ is of weight divisible by four unless otherwise stated.
This implies that $u$ generates a type II code.
It is in general easier to calculate the minimum distance of type II codes, as opposed to their singly even type I counterparts, due to the extra restriction on the possible weights of codewords.
As in the previous sections the code $\mathcal{C}$ is the principal left ideal of $u$.
Again the matrix $G = [I|A]$ is a generator matrix for the vector form of $\mathcal{C}$ consisting of the first $k$ rows of $U$, $u$'s group ring matrix according to the usual listing $\{1,b,b^2,\ldots,b^{k-1},a,ab,ab^2,\ldots,ab^{k-1}\}$ of $\mathbf{D}_{2k}$.

The non-zero codewords of a type II code are each of weight divisible by four.
Thus the only possible weights of non-zero codewords are four, eight, sixteen, and so on.
The minimum distance of a linear code (which $\mathcal{C}$ is since $\mathbb{Z}_2$ is a field) is equal to the minimum of the weights of its non-zero codewords~\cite[p.~8]{huf03}.
Thus the minimum distance of a non-zero type II code is a multiple of four.

To show that such a code's minimum distance is exactly $4l$ for some positive integer $l$ we need only show that no codeword is of weight $4m$ for $m$ an integer from one to $l-1$.
In the case of the code $\mathcal{C}$ we will now show that no codeword is of weight four so long as the first row and $k$ different two-row combinations of its generator matrix $G=[I|A]$ are not.
We begin by discussing the fact that no combination of five or more rows of $G$ can be of weight four.

\subsection{Combinations of Five or More Rows}
We can easily see that every combination of five or more rows of $G$ is of weight at least weight eight.
The subvector containing the first $k$ components of a combination of $i$ rows of $G$ is of weight exactly $i$.
This is due to the fact that it is the combination of $i$ rows of the identity matrix.
Hence every combination of five or more rows of $G$ is of weight at least five.
Thus such combinations of rows of $G$ are not of weight four and must be of weight at least eight.
The only way for a codeword can be of weight less than eight is if it is the combination of four or less rows of $G$.
In the next section we see that every combination of four rows is of weight at least eight.

\subsection{Combinations of Four Rows}
\label{sect:gencombsfourrows}
The subvector consisting of the first $k$ components of the combination of four rows of $G$ is of weight four.
For such a combination to be of weight exactly four the subvector consisting of the last $k$ components of the combination must thus be of weight zero.
We will now see that the subvector consisting of the last $k$ components is of weight at least one.
We show this by proving that the rows of the matrix $A$ are linearly independent.

The rows of $A$ are linearly independent because the group ring matrix $U$ when squared is equal to zero.
Working block-wise:
\begin{equation*}
\begin{split}
&\qquad U^2 = \underline{0}_{2k} \\
&\Leftrightarrow \left[ \begin{array}{cc} I & A \\ A & I \end{array} \right] \left[ \begin{array}{cc} I & A \\ A & I \end{array} \right] = \underline{0}_{2k} \\
&\Leftrightarrow \left[ \begin{array}{cc} I + A^2 & A + A \\ A + A & I + A^2 \end{array} \right]  = \left[ \begin{array}{cc} \underline{0}_{k} & \underline{0}_{k} \\ \underline{0}_{k} & \underline{0}_{k} \end{array} \right] \\
&\Leftrightarrow \left[ \begin{array}{cc} I + A^2 & \underline{0}_{k} \\ \underline{0}_{k} & I + A^2 \end{array} \right]  = \left[ \begin{array}{cc} \underline{0}_{k} & \underline{0}_{k} \\ \underline{0}_{k} & \underline{0}_{k} \end{array} \right]. \\
\end{split}
\end{equation*}
Thus $U^2$ is equal to zero if and only if $A^2$ is equal to the identity matrix.

This implies that $A$ is its own inverse and matrices with inverses are of full rank.
The rows of $A$ are therefore linearly independent and so combinations of four of these are non-zero.
Thus each combination of four rows of $G$ is not of weight four.
Hence the combinations of four rows of $G$ are each of weight at least eight.

We are now in the position that no non-zero combination of rows of $G$ is of weight less than eight unless some non-zero combination of three or less rows of $G$ is.
We will see in the next section that a three-row combination of $G$ can only be of weight four if row one of $G$ is.

\subsection{One and Three Rows}
\label{sect:genonethreerows}
We consider now the combinations of three rows of $G$ and their relation to the single rows of $G$.
Such combinations are of weight at least three due to the identity matrix in $G$.
Consider the possibility that such a three-row combination of $G$ is of weight four.
Since the subvector of the first $k$ components of the codeword are of weight exactly three, that of the last $k$ must be of weight one.

In section \ref{sect:genquasicyclic} we showed that $\mathcal{C}$ as generated by $G$ is quasi cyclic of index two.
Suppose the vector $\underline{c} = \underline{a} \underline{b}$ is a combination of three rows of $G$ that is of weight four.
The result, $\underline{d} = \underline{b} \underline{a}$, of switching the first $k$ components with the last $k$ is also a codeword, and is of the same weight as $\underline{c}$.
The subvector $\underline{b}$ is of weight one and is therefore, in $\underline{d}$, the result of a linear combination of a single row of the identity matrix by itself.
Thus $\underline{d}$ is one of the rows of $G$.
A combination of three rows of $G$ can only therefore be of weight four if one of the rows of $G$ is of weight four.

Every row of $G$ is a permutation of the first row and hence has the same weight as this first row.
The weight of a single row of $G$ is therefore only of weight less than eight if the first row is.
When it is not, the only combinations of rows of $G$ that could possibly be of weight less than eight are those of two-row combinations.
These are discussed in the next section.

\subsection{Two Rows}
We now discuss the possibility that a two-row combination of $G$ is of weight four.
If no such combination is of weight four then, provided that the first row of $G$ is not of weight four, the code $\mathcal{C}$ is of minimum distance at least eight.
In the following we prove that each combination of two rows of $G$ is equal in weight to at least one of the combinations of row one with one of the rows two to $k/2 + 1$ of $G$.

Let $r_{i+1}$ be the $(i+1)^{\textrm{th}}$ row of $G$ for $i$ from zero to $k-1$.
Recall from section \ref{sect:vectorforms} that the group ring codeword corresponding to $r_{i+1}$ is $b^i u$.
Thus the combination of two distinct rows, $r_{j+1}$ and $r_{k+1}$, of $G$ for $0 \leq j < k \leq k-1$, corresponds to the group ring codeword $b^j u + b^k u$.
Multiplying this group ring codeword by the inverse $b^{-j}$ of $b^j$ will not affect its weight.
Thus it is equal in weight to the group ring codeword $u + b^{k-j} u$.
In vector form this codeword is the combination of rows one and $(k-j)+1$ of $G$, where the calculation $(k-j)$ is done modulo $k$ in accordance with the group multiplication.
Thus the combination of two rows of $G$ is equal to at least one combination of row one of $G$ with one of the other rows of $G$.

We could likewise have multiplied the original group ring codeword $b^j u + b^k u$ by $b^{-k}$, leaving its weight intact.
This would result in the codeword that is the group ring form of the combination of rows one and $(j-k)+1$ of $G$.
Thus the combination of rows $j$ and $k$ of $G$ is in fact equal in weight to the two combinations of row one of $G$ with each of rows $(j-k)+1$ and $(k-j)+1$.
Modulo $k$ the two numbers $j-k$ and $k-j$ are both non-zero but sum together to give zero.
Therefore either they are both $k/2$ or one is congruent to an integer between one and $(k/2)-1$ inclusive.
At least one of the rows $(j-k)+1$ and $(k-j)+1$ is thus one of rows two to $k/2+1$ of $G$.

To check that every combination of two rows of $G$ is not of weight four we therefore need only check that the combinations of row one with each of rows two to $k/2 + 1$ are not.
There are $k/2$ such combinations.
We saw in section \ref{sect:genonethreerows} that all of the other non-zero combinations of $G$ are of weight at least eight if the first row of $G$ is.
Therefore to prove that a code $\mathcal{C}$ is of minimum distance at least eight we need only check $k/2 +1$ different combinations of rows of the generator matrix $G$.
This is quite a nice result as the total number of codewords in such a code $\mathcal{C}$ is $2^k$.

Note that the fact that we need only check the weight of these given $k/2$ two-row combinations of $G$ can be extended to checking that the dot product of distinct rows of $A$ is zero.
In section \ref{sect:gencombsfourrows} we showed that $u^2$ was equal to zero if and only if $A^2$ was equal to the identity matrix.
Since $A$ is symmetric\footnote{A consequence of it being reverse circulant.} it is equal to zero when squared if and only if the dot product of each pair of distinct rows is zero and that of each row with itself is one.
We can easily adapt the above proof to show that the dot product of each pair of distinct rows of $A$ is zero if and only if that of row one with each of rows two to $k/2+1$ is.
The generator $u$ is of weight divisible by four and so $\mathbf{f}$ is of odd weight and so is each row of $A$.
The dot product of each row of $A$ with itself is thus one.
These observations are useful when searching for zero divisors of the form $u = 1 + \mathbf{f}$ with $u^2$ equal to zero.

The idea that the minimum distance of the code $\mathcal{C}$ is at least eight works quite nicely on zero divisors that generate codes of length less than forty-eight.
In the next section we show that, using the techniques just discussed, we can find generator for extremal type II codes for every possible length less than forty-eight.

\section{Up To Length Forty-Eight}
The highest minimum distance a type II code of length less than twenty-four can have is four~\cite[p.~346]{huf03}.
The maximum a type II code of length at least twenty-four and up to length forty-eight can have is eight~\cite[p.~346]{huf03}.
Codes achieving these minimum distances are termed `extremal'~\cite[p.~346]{huf03}.
We have been able to find zero divisors in the relevant group rings that generate extremal codes of every length under forty-eight.
We used the programs listed in appendix \ref{appen:programs} to find them.
Remember that type II codes are only ever a multiple of eight.
The proofs in the previous sections show that they are principal ideals in their dihedral group rings, self-dual, doubly even, quasi cyclic of index two, generated by reverse circulant generator matrices and that they are of their claimed minimum distance.

We begin with the $(8,4,4)$ extended Hamming code, which was discussed in section \ref{sect:exthamm}.
The code is of length eight so the dihedral group ring in question is $\mathbb{Z}_2 \mathbf{D}_8$, that of the finite field with two elements $\mathbb{Z}_2$ and the dihedral group with eight elements $\mathbf{D}_8 = \langle a , b \mid a^2 , b^4 , (ab)^2 \rangle$.
The only elements of the form $1 + a \mathbf{f}$, where $\mathbf{f}$ is sum of powers of the generator $b$, in this group ring that are of weight a multiple of four are then $1 + a (1 + b + b^2)$, $1 + a ( 1 + b^2 + b^3 )$, $1 + a ( 1 + b + b^3)$ and $1 + a ( b + b^2 + b^3 )$.
Multiplying any of these elements by themselves gives zero.
Hence they all generate self-dual codes that are doubly even.
Since such a code has minimum distance a multiple of four, the minimum distance of the codes generated by each of these is four.

The next length of code in the series of extremal type II codes is sixteen.
Again, type II codes of length sixteen have minimum distance at most four.
We found sixteen elements in $\mathbb{Z}_2 \mathbf{D}_{16}$ that generate such a code.
We denote $\mathbf{D}_{16}$ as $\langle a , b \mid a^2 , b^8 , (ab)^2 \rangle$.
They are the elements $1 + a b^i ( 1 + b^2 + b^4)$ and $1 + a b^i ( 1 + b + b^2 + b^3 + b^4 + b^5 + b^6)$ for $i$ from zero to seven.
Note that in section \ref{sect:withouttranspose} we showed that then so should the elements $1 + a b^j (1 + b^2 + b^4)^{\textrm{T}}$ and $1 + a b^j ( 1 + b + b^2 + b^3 + b^4 + b^5 + b^6)^{\textrm{T}}$ for $j$ from zero to seven generate the code.
This is indeed the case, however those elements are all equal to one of the former types of elements, giving only sixteen unique generators.

We now move on to the case of length twenty-four.
At length twenty-four the upper bound $(4\lfloor n/24 \rfloor + 4)$ on the minimum distance of a type II code increases from four to eight~\cite[p.~346]{huf03}.
The extended binary Golay code is the unique extremal type II code of this length~\cite[p.~401]{huf03}.
We showed in chapter \ref{chap:extgolaycode} that it is generated by twenty-four different elements.
These are the elements $1+a\mathbf{f} = 1 + a(b + b^2 + b^4 + b^5 + b^6 + b^7 + b^9)$, $1+ab^i\mathbf{f}$ for $i$ from one to eleven and $1+ab^j\mathbf{f}^{\textrm{T}}$ for $j$ from zero to eleven.

The next length that is a multiple of eight is thirty-two.
We searched for elements of the form $1 + a \mathbf{f}$ in the group ring $\mathbb{Z}_2 \mathbf{D}_{32}$ that generated codes of minimum distance eight.
We found one hundred and twenty-eight generators of codes of minimum distance eight.
Half of these are the elements $1 + a b^i ( 1 + b^3 + b^{10} + b^{11} + b^{12} + b^{13} + b^{14})$, $1 + a b^i ( 1 + b^5 + b^6 + b^{10} + b^{11} + b^{12} + b^{13} )$, $1 + a b^i ( 1 + b^2 + b^5 + b^7 + b^{12} + b^{13} + b^{14})$ and $1 + a b^i ( 1 + b^4 + b^5 + b^7 + b^{10} + b^{13} + b^{14})$ for $i$ from zero to sixteen.
The other half are same elements with the $\mathbf{f}$ part transposed.

The only length left before forty-eight is forty.
The group ring is then $\mathbb{Z}_2 \mathbf{D}_{40}$.
We found nine hundred and twenty $(40,20,8)$ type II code generating elements of the form $1 + a \mathbf{f}$ in the group ring.
Some are of weight seven and some of weight eleven.
An example of a weight seven generator is $1 + a ( 1 + b + b^3 + b^4 + b^{14} + b^{15} + b^{16} )$ and that of a weight eleven generator is $1 + a ( 1 + b^2 + b^4 + b^7 + b^9 + b^{10} + b^{12} + b^{13} + b^{16} + b^{17} + b^{18})$.
An interesting topic for further research would be to asses the equivalence of the codes generated by these elements.

The upper bound on the minimum distance of codes of length forty-eight and above is greater than eight.
Unfortunately we have not been able to find nice quick results to prove that the minimum distance of such a type II code is larger than eight.
To prove that the minimum distance is twelve for instance, we must prove that no combination of rows of $G$ is of weight four or eight, as opposed to just four.
The proof we used in the last chapter to prove that the minimum distance of the $(48,24,12)$ type II code is twelve was complex and relied on the specific properties of the zero divisor used to construct the code.
It can therefore not be easily adapted to the general case.
We can however, use some of the properties of the construction to vastly reduce the number of combinations of rows of $G$ that must be checked to verify the code is of a general minimum distance.
We discuss these properties and their uses in the next two sections.

\section{Dih-Cycling}
In the previous chapter we discussed the idea of dih-cycling three-row and four-row combinations of the basis elements given there for the $(48,24,12)$ type II code.
Here in this section we discuss the fact that dih-cycling works for any code of the form we have been talking about.
Let $u = 1 + a \mathbf{f}$ be an element in the group ring $\mathbb{Z}_2 \mathbf{D}_{2k}$.
As we have been doing all along, we let $\mathbf{D}_{2k} = \langle a , b \mid a^2 , b^k , (ab)^2 \rangle $ and then $\mathbf{f}$ is a sum of powers of $b$.
Furthermore we assume that $u^2$ is equal to zero.
The first $k$ rows of the group ring matrix of $u$ according to the listing $\{1,b,b^2,\ldots,b^{k-1},a,ab,ab^2,\ldots,ab^{k-1}\}$ of $\mathbf{D}_{2k}$ then form a generator matrix for the code $\mathcal{C} = (\mathbb{Z}_2 \mathbf{D}_{2k})u$.
These $k$ rows correspond to the group ring elements $u , b u , b^2 u , \ldots , b^{k-1} u$ as discussed in section \ref{sect:vectorforms}.
These elements thus form a basis for the group ring code $\mathcal{C} = (\mathbb{Z}_2 \mathbf{D}_{2k})u$ over $\mathbb{Z}_2$.

Any combination of those basis elements is equal in weight to at least one involving $u$.
Let $b^{i_1} u + b^{i_2} u + \ldots + b^{i_l} u$ be such a combination of $l$ elements for $l$ a positive integer with $1 \leq l < k$.
We can multiply this combination by the group ring element by $b^{-i_j}$ for any $j$ and it is still a combination of basis elements.
In particular one of the basis elements in the new combination is $u$.
Thus each combination of $l$ basis elements is equal in weight to a combination of $l$ basis elements where one of those is $u$.
In the next section we discuss how dih-cycling can reduce the calculations in the determination of the minimum distance of a dihedral code.

\section{Minimum Distance Checking}
\label{sect:genmindistchecking}
We use the same notation here as in the last section.
The code $\mathcal{C}$ is the span of the elements $u , b u , b^2 u , \ldots , b^{k-1} u$.
An arbitrary codeword is a combination $b^{i_1} u + b^{i_2} u + \ldots + b^{i_l} u$ for $i_j$ an integer for all $j$.
To check that the minimum distance of the code $\mathcal{C}$ is a given value $d$ we check that every non-zero combination of them is of weight at least $d$~\cite[p.~8]{huf03}.
To do this we can just check those combinations involving $u$, since as we saw in the last section we can dih-cycle any combination to become one involving $u$ leaving its weight intact.
Note that dih-cycling does not change the number of basis elements with non-zero coefficient in a combination.

Of course, we can dih-cycle according to any of the indices $i_1$, $i_2$, $\ldots$, $i_l$ in the combination $b^{i_1} u + b^{i_2} u + \ldots + b^{i_l} u$ and arrive at a combination of basis elements involving $u$.
We can thus, out of all these dih-cycled combinations of equal weight, pick one such combination and check only its weight.
Consider the quantities $i_k - i_j$ modulo $k$ for all $1 \leq j < k \leq l$.
At least one of these has the property that it is less than or equal to all of the others.
Since we are free to dih-cycle any of the basis elements in a combination until it becomes $u$, we can therefore choose to check only the weights of the combinations $u + b^{m_1} u + \ldots + b^{m_{l-1}} u $ where $m_1 \leq ( m_k - m_j )$ for all $k$ and $j$.

Remember of course that this dih-cycling technique is applied after many of the $2^k$ combinations of the basis elements have already been ruled out as possibly having low minimum distance.
Every combination of $l$ such basis elements has weight at least $l$ since the first $k$ components of the first $k$ rows of the group ring matrix $U$ is the identity matrix.
Thus to show that the minimum distance is some multiple of four $d$ we need only show that every combination of less than $d-3$ rows is of weight at least $d$.
Furthermore no set of $d-4$ rows can be of weight $d-4$ as we showed in section \ref{sect:gencombsfourrows}.
The quasi cyclic nature of the code also reduces the possibilities.
We saw in section \ref{sect:genonethreerows} that if for instance three of the rows combine to a vector of weight four then at least one row is of weight four.

These techniques are extremely useful in dealing with large dihedral codes.
As examples we now discuss two codes, one of length seventy-two and one of length ninety-six.

\section{A $(72,36,12)$ Code}
\label{sect:gen723612}
As discussed in the introduction to this chapter the proof or disproof of the existence of a putative $(72,36,16)$ type II code is a long-standing open problem in coding theory.
We performed an exhaustive search for zero divisors of the form $1 + a \mathbf{f}$ in $\mathbb{Z}_2 \mathbf{D}_{72}$ that generate codes of minimum distance sixteen.
As before, the zero divisors we searched had the property that $u^2$ was equal to zero and $\mathbf{f}$ was a sum of powers of the generator $b$ where $\mathbf{D}_{72} = \langle a , b \mid a^2 , b^{36} , (ab)^2 \rangle $.
No such zero divisors were found.
We did however discover zero divisors that generate type II $(72,36,12)$ codes.
Twelve is the best known minimum distance for such a type II code of length seventy-two to date.

We found one thousand, seven hundred and sixty-four such zero divisors of the form $1 + a \mathbf{f}$.
Two examples of these are $1 + a(1 + b + + b^2 + b^5 + b^6 + b^7 + b^8 + b^{10} + b^{11} + b^{12} + b^{14} + b^{15} + b^{16} + b^{17} + b^{19} + b^{22} + b^{23} + b^{25} + b^{26} + b^{27} + b^{28} + b^{29} + b^{30} )$ and $1 + a( 1 + b + b^2 + b^5 + b^6 + b^7 + b^8 + b^{10} + b^{11} + b^{13} + b^{14} + b^{15} + b^{17} + b^{18} + b^{19} + b^{21} + b^{22} + b^{23} + b^{25} + b^{26} + b^{27} + b^{28} + b^{29})$.
As discussed in section \ref{sect:mindisteight}, it is easy to show that these elements generate codes of minimum distance at least eight.
A computer check, perhaps using the techniques in section \ref{sect:genmindistchecking}, will convince the reader that these elements actually generate type II codes of minimum distance twelve.
In the next section we discuss a longer code, of length ninety-six, which also has the best known minimum distance for its type and length.

\section{A $(96,48,16)$ Code}
\label{sect:gen964816}
As well as searching for a $(72,36,16)$ type II code, we also conducted a search for a $(96,48,20)$ type II code.
Again the existence of such a code is a long-standing open problem.
We were not able to find a generator of the form $1 + a \mathbf{f}$ for such a code, in $\mathbb{Z}_2 \mathbf{D}_{96}$.
The search was not exhaustive however, as the search space is extensive and beyond the ability of our algorithms and computers.

We were able to find generators of $(96,48,16)$ type II codes, however.
Two of these are the zero divisors $1 + a( 1 + b^2 + b^4 + b^5 + b^6 + b^7 + b^8 + b^{10} + b^{11} + b^{14} + b^{15} + b^{17} + b^{18} + b^{22} + b^{23} + b^{24} + b^{25} + b^{26} + b^{31} + b^{33} + b^{36} + b^{38} + b^{39})$ and $1 + a ( 1 + b + b^3 + b^6 + b^8 + b^{13} + b^{14} + b^{15} + b^{16} + b^{17} + b^{21} + b^{22} + b^{24} + b^{25} + b^{28} + b^{29} + b^{31} + b^{32} + b^{33} + b^{34} + b^{35} + b^{37} + b^{39} )$.
Again the techniques in section \ref{sect:genmindistchecking} can be employed to show these generate codes of minimum distance sixteen.

Length ninety-six is at the very limit of the abilities of our computers to date.
Further research into the algebraic nature of the generators might help to discover why the generators we have found generate the codes they do.
In chapter \ref{chap:extgolaycode} we discussed some properties unique to the generators of the extended binary Golay code in their group ring.
In the next section we discuss these properties further and in terms of the zero divisors listed above for the longer codes.

\section{Properties of the Generators}
\label{sect:genproperties}
The second part of chapter \ref{chap:extgolaycode} was dedicated to showing that the twenty-four zero divisors of the form $1+ a \mathbf{f}$ listed there were the only ones of that form in the group ring $\mathbb{Z}_2 \mathbf{D}_{24}$ that could possibly generate the extended binary Golay code.
The reason behind pursuing such a result was the potential of finding out what made those generating elements work.
Perhaps in the future we will discover general properties of zero divisors of this form in dihedral group rings that always generate codes that are of good minimum distance.

A unique property we discovered to the generators of the extended binary Golay code was that their multiset of differences was of the form $(\mathbf{10} \times 4) + (\mathbf{1} \times 2)$.
Remember that this means that ten of the non-identity powers of the group element $b$ appear four times and one of them appears twice in the multiset of differences $\mathcal{D} = \{ b^{k-l} \mid b^k , b^l \in \mathbf{f} , k \neq l \}$ of $\mathbf{f}$ where the generator is $u = 1 + a \mathbf{f}$.
This property was sufficient to guarantee that the zero divisor generated the code.

We also, in the previous chapter, discussed the $(48,24,12)$ type II code.
All of the zero divisors in $\mathbb{Z}_2 \mathbf{D}_{48}$ that are listed there have multisets of differences of the form\footnote{The form of a multiset of differences is defined in section \ref{sect:multidiffs}.} $( \mathbf{13} \times 10 ) + ( \mathbf{10} \times 8)$.
We have found this form to not be unique to those generators however.
Some zero divisors of the form $1 + a \mathbf{f}$ have a mutliset of differences of that form and do not generate a code of minimum distance twelve.
In our efforts to prove algebraically, as we did, that the minimum distance of the code is twelve we discovered that the zero divisors that do generate a code of minimum distance twelve all had generator matrices of the form $G = [I|A]$ where rows one, nine and seventeen have exactly six $1$-components in common.
These three rows are evenly spaced throughout the reverse circulant generator matrix.
Along with the multiset of differences, this property appears sufficient to lead to the generation of the $(48,24,12)$ type II code.
This eventually led to our discovery that the group ring listing in which the group elements corresponding to those three rows come first enabled us to prove the minimum distance of the code.

In the length seventy-two case however, two different values are allowed for the number of $1$-components shared between three evenly spaced rows throughout the generator matrix.
It seems that these rows can share either six or nine $1$-components and still manage to generate a minimum distance twelve code.
All of those with nine shared generate a code of minimum distance twelve, but some of those with six shared do not.
Further research into the sets of differences and the numbers of $1$-components shared between three rows evenly spaced throughout the generator matrices of good dihedral codes may lead to the ability to generate good codes of longer length using the same techniques as given in this thesis.
Another interesting topic for further research would be the investigation of other group rings in which it is possible to find generators for the same codes, as discussed in the next section.

\section{Using Other Regular Subgroups}
\label{sect:otherregsubgroups}
One of the reviewers of our publication of our construction of the extended binary Golay code~\cite{mcl08} referred to the fact that the dihedral group is a regular subgroup of the automorphism group of the code.
The automorphism group of the extended binary Golay code is the Mathieu group on twenty-four points $\mathcal{M}_{24}$~\cite[p.~251]{huf03}.
The reviewer conjectured that, having found generators for the code in the dihedral group ring, we should be able to find generators for the code in group rings using any of the regular subgroups of the automorphism group of the code.
He was correct.

Using some of the GAP code listed in appendix \ref{appen:programs} we were able to find generators of the code in the group rings composed of the finite field with two elements $\mathbb{Z}_2$ and each of the following groups: $\mathbf{D}_{24}$; $\mathbf{C}_2 \times \mathbf{A}_4$, the direct product of the cyclic group of order two and the alternating group of order twelve; $\mathbf{S}_4$, the symmetric group of degree four; $\mathbf{C}_3 \times \mathbf{D}_8$, the direct product of the cyclic group of order three and the dihedral group of order eight; and $( \mathbf{D}_8 \times \mathbf{C}_3) $ the direct product of the dihedral group of order eight with the cyclic group of order three.
These are all of the regular subgroups of $\mathcal{M}_{24}$.
Investigation of the nature of the connection between generators of the same code in different group rings would be an extremely interesting topic for further study.
For more on regular subgroups of the automorphism groups of group ring matrices the reader is referred to Horadam's book on the subject of Hadamard matrices~\cite{hor07}.
This concludes our discussion of general dihedral codes.